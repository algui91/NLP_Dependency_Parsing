%*******************************************************
% Abstract
%*******************************************************
%\renewcommand{\abstractname}{Abstract}
\pdfbookmark[1]{Resumen}{Resumen}
\begingroup
\let\clearpage\relax
\let\cleardoublepage\relax
\let\cleardoublepage\relax

\chapter*{Resumen}
En este trabajo se implementa un método para analizar dependencias palabra a
palabra con una estrategia de abajo a arriba (\textit{Bottom-Up}) usando
\acfi{SVM}. En concreto, este trabajo se ha centrado en analizar las
dependencias entre palabras en Castellano. El algoritmo no usa información sobre
la estructura de las frases, realiza un análisis determinístico de las
dependencias del idioma. El funcionamiento consiste en ir generando un árbol de
dependencias para la frase mediante la ejecución de tres acciones ---
\textsc{Desplazar, Izquierda, Derecha} --- Inicialmente se comienza con tantos
nodos como palabras tiene la frase, a medida que se ejecuta el algoritmo, se
aplican las acciones mencionadas para ir construyendo el árbol. Por ejemplo, al
aplicar la acción \textsc{Derecha} a dos nodos contiguos, el nodo posicionado a
la derecha pasa a ser padre del nodo izquierdo. En cuanto a los resultados
obtenidos, han sido coherentes a la descripción proporcionada por los autores,
obteniéndose resultados muy similares, incluso ligeramente mejores. La
implementación original -- \citet{yamada2003} -- no se basa en conjuntos de
datos en castellano, de modo que las comparaciones se han realizado frente al
trabajo de \citet{rohit2016}.

\paragraph{Etiquetas:} PNL, SVM, Parseo de dependencias.

\vfill

\begin{otherlanguage}{american}
  \pdfbookmark[1]{Abstract}{Abstract}
  \chapter*{Abstract}
  In this project, a method for analyzing word to word dependencies is
  implemented using a bottom-up strategy with the help of \acf{SVM}. In
  particular, this work has focused in analyzing dependencies between words for
  spanish language. The algorithm does not use any information about sentence's
  structure, it performs a deterministic analysis of the language's
  dependencies. Dependencies are build by generating a dependency tree for the
  sentence being analized. This tree is build with the help of three parsing
  actions --- \textsc{Shift, Left, Right} --- At the beginning the algorithm has
  as many nodes as words has the sentence, in each iteration an action is
  applied to the nodes in order to build the dependency tree. For instance, when
  \textsc{Right} action is applied to a pair of nodes, the node on the right
  becomes the parent of the left node. The accuracy obtained by this project its
  coherent with the one obtained for other authors. Our accuracy its very
  similar, even a little bit better. The original implementation by
  \citet{yamada2003} did not use any Spanish datasets for its experiments, so
  comparison has been made against the work of \citet{rohit2016}.
  
  \paragraph{Tags:} NLP, SVM, Dependency parsing.
\end{otherlanguage}

\endgroup			

\vfill