%*****************************************
\chapter{Parseo de dependencias en Español}\label{ch:depparsing}
% *****************************************

\todo[inline]{Desribir de forma más detallada el Dep Parsing, mencionar state of
  the art (3/4 papers), entre ellos el implementado}

\todo[inline]{Sección con algoritmo implementado, reiterando sección anterior
  pero con lujo de detalles (Teóricos y código)}

% Los titulas como te digo y extiendes un poco lo que se hace en cada uno de ellos
% (3-4 frases por objetivo):

% - Revisión bibliográfica del estado del arte y antecedentes del dep parsing en
% español. (Enlazas con capitulos de la memoria de antecedentes y dep parsing)

% - Elección y análisis de requerimientos de un procedimiento apropiado de dep
% parsing a partir de la revisión anterior y diseño del mismo para el lenguaje de
% programación Scala (Enlazas con la explicación teórica del algoritmo que has
% implementado y con un capítulo de introducción a Scala).

% - Implementación y procesos de prueba del algoritmo escogido. (Capítulos de
% implementación del algoritmo y de tests de ingeniería del software)

% - Evaluación del algoritmo, comparación y discusión de resultados obtenidos en
% casos prácticos (Capítulo sobre resultados obtenidos, comparación con los
% resultados originales y otros disponibles y último capítulo de conclusiones).

% Los objetivos mencionan la resolución del trabajo como uno de ellos, pero sin
% entrar en detalle tal y como en el que se entra en los capítulos
% correspondientes. La sección de objetivos se limita a enumerarlos y explicar
% brevemente en qué consisten. En la Resolución del trabajo, te explicas con todo
% lujo de detalles técnicos y científicos sobre los métodos desarrollados.

A continuación se listan los objetivos previstos del trabajo.

\paragraph{Revisión bibliográfica del estado del arte y antecedentes del parseo
  de dependencias en Español.} Este objetivo pretende explorar los métodos
existentes que realizan parseado de dependencias, para adquirir un conocimiento
previo del \nameref{sec:stateoftheart} en la literatura. Así como conocer los
métodos existentes para el parseo de dependencias en Español
\cite{ballesteros2016}.

\paragraph{Elección y análisis de requerimientos de un procedimiento apropiado de
  parseo de dependencias y diseño para Scala}
Debido a la popularidad del lenguaje de programación \textsc{Scala} en el área
del \ac{AA} y la minería de datos, se pretende implementar este trabajo bajo
dicho lenguaje.\myTodo{Enlazar con explicación algoritmo implementado, y
  capítulo de intro a \textsc{scala}}. El algoritmo de parseo de dependencias
elegido se ha basado en \citeauthor{yamada2003} \cite{yamada2003} y
\citeauthor{rohit2016} \cite{rohit2016}.

\paragraph{Implementación y procesos de prueba del algoritmo escogido}



\paragraph{Evaluación del algoritmo, comparación y discusión de resultados
  obtenidos en casos prácticos}


%*****************************************
%*****************************************
%*****************************************
%*****************************************
%*****************************************
