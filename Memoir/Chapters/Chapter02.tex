%*****************************************
\chapter{Parseo de dependencias en Español}\label{ch:depparsing}
% *****************************************

A continuación se listan los objetivos previstos del trabajo.

\paragraph{Revisión bibliográfica del estado del arte y antecedentes del parseo
  de dependencias en Español.} Este objetivo pretende explorar los métodos
existentes que realizan parseado de dependencias, para adquirir un conocimiento
previo del \nameref{sec:stateoftheart} en la literatura. Así como conocer los
métodos existentes para el parseo de dependencias en Español
\cite{ballesteros2016}.

\paragraph{Elección y análisis de requerimientos de un procedimiento apropiado de
  parseo de dependencias y diseño para Scala}
Debido a la popularidad del lenguaje de programación \textsc{Scala} en el área
del \ac{AA} y la minería de datos, se pretende implementar este trabajo bajo
dicho lenguaje. El algoritmo de parseo de dependencias elegido se ha basado en
\citeauthor{yamada2003} \cite{yamada2003} y \citeauthor{rohit2016}
\cite{rohit2016}. Una introducción al lenguaje de programación \textsc{Scala}
puede consultarse en el \autoref{ch:scalaintro}. La descripción del algoritmo
propuesto por \citeauthor{yamada2003} se detalla en el \autoref{ch:algorithm}.

\paragraph{Implementación y procesos de prueba del algoritmo escogido}
Aquí se implementará el algoritmo escogido para el parseado de dependencias. Así
mismo, se pondrá en práctica la técnica de desarrollo al estilo
\acfi{TDD}\acused{TDD}, en concreto se usará \acfi{BDD}\acused{BDD} orientado a
problemas de \ac{AA}. La implementación se encuentra en el \autoref{ch:impl},
mientras que las pruebas pertinentes pueden encontrarse en el \autoref{ch:tests}

\paragraph{Evaluación del algoritmo, comparación y discusión de resultados
  obtenidos en casos prácticos}
Con este objetivo se pretende mostrar los resultados obtenidos con la
implementación realizada, así como una comparación de los resultados en la
implementación original. Estos resultados se dicutirán en la \myTodo{Enlazar a
  sección de resultados y conclusiones}sección x.

%*****************************************
%*****************************************
%*****************************************
%*****************************************
%*****************************************
