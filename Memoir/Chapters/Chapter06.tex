%************************************************
\chapter{Casos de prueba orientados a Aprendizaje Automático}
\label{ch:tdml}

En este capítulo se verán los casos de prueba implementados para el proyecto. Se
ha seguido un estilo de \emph{tests} orientado a problemas de \ac{AA}. En las próximas
secciones se introduce este tipo de práctica.

%************************************************
\section{Test-Driven Development}
\label{sec:tdd}

El \acfi{TDD} se basa en dos principios muy claros \cite{Justin2015}:
\begin{itemize}
\item No escribir ninguna línea de código nueva a no ser que se tenga un
  \emph{test} fallido.
\item Eliminar duplicidades.
\end{itemize}

En esencia, \ac{TDD} es un proceso en el desarrollo de \emph{software} que
permite al desarrollador escribir código que especifica el comportamiento que
poseerá el programa antes de que este sea implementado. La ventaja de este
estilo de desarrollo reside en que a cada paso que se avanza, se obtiene un
\emph{software} completamente funcional, así como el conjunto de
especificaciones que lo definen. En el \ac{TDD} está inherente en cada momento
el desarrollo mediante prueba y error, al igual que en el \ac{AA}.

El proceso al que nos sometemos al adoptar la filosofía \ac{TDD} cambia la forma
en la que se piensa al desarrollar código. Además, el \emph{software} diseñado
como resultado será mucho más modular, permitiendo tener distintos componentes
que se pueden intercambiar en todo el \emph{pipeline}.

Cuando se escribe de antemano la intención del código, antes de implementarlo de
verdad, se aplica una presión al diseño del mismo que evita escribir código del
llamado ``\emph{Por si acaso}''. Con el uso de \ac{TDD}, primero se piensa en un
caso de prueba, se ve que el \emph{software} aún no lo soporta y entonces se
corrige. Si no se es capáz de pensar en un caso de prueba, no se añade código.

\ac{TDD} opera a varios niveles. Los \emph{tests} pueden escribirse para
funciones, métodos, clases, programas, servícios webs, redes neuronales y
\emph{pipelines} de \ac{AA} al completo. En todo momento, independientemente del
nivel, los \emph{tests} se escriben desde la perspectiva del cliente. En este
proyecto se ha centrado el tipo de \emph{tests} hacia el \ac{AA}. En este
contexto, los \emph{tests} se escriben para funciones, métodos, clases,
implementaciones matemáticas y todos los algoritmos de aprendizaje.

\subsection{El ciclo de \ac{TDD}}
\label{subsec:tddcycle}

El ciclo para \ac{TDD} consiste en escribir trozos pequeños de funciones que
intenten hacer algo que aún no está implementado. Normalmente se suele
estructurar el \emph{test} en tres partes principales. En la primera se preparan
los objetos y datos necesarios. En la segunda se llama al código para el cual se
está escribiendo el \emph{test}. Por último, se valida si el resultado del
código es el que se esperaba. En la primera fase se escribe el código
necesario para hacer pasar el \emph{test} --- en este momento no es relevante
que el código sea correcto o esté bien diseñado, el único objetivo es hacer
pasar el \emph{test}. --- Una vez se tiene el \emph{test} correcto, se procede a
refactorizar el código. Cabe destacar que en este contexto \emph{refactorizar}
significa cambiar la forma en la que se escribió el código, pero bajo ningún
concepto cambiar cómo se comporta.

Por tanto, el ciclo del \ac{TDD} se puede dividir en tres pasos: \textsc{Rojo},
\textsc{Verde} y \textsc{Refactorizar}.

\subsubsection{Rojo}
\label{sec:tddred}

Primero se crea un \emph{test} que no funciona. Al más alto nivel en cuanto a
\ac{AA} se refiere, podría ser un punto de partida como conseguir un resultado
mejor que una predicción aleatoria. El punto de partida puede ser el que se
quiera, el anterior es solo un ejemplo, otros podrían ser: \emph{Predecir
  siempre lo mismo}.

\subsubsection{Verde}
\label{sec:tddgreen}

Una vez se tiene un \emph{test} que está fallando, se procede a arreglarlo, de
ahí el nombre de este paso -- \textsc{Verde}. -- Si se ha empezado con un
\emph{test} a un nivel de abstracción demasiado elevado, se puede dividir el
mismo en varios a más bajo nivel y comenzar a arreglarlos. El objetivo aquí es
hacer que el \emph{test} pase lo antes posible, es por ello que se permite
programar cualquier cosa con tal de obtener un \emph{test} ``verde'', es en la
siguiente fase donde se procede a refactorizar.

\subsubsection{Refactorizar}
\label{sec:tddrefactor}

Una vez se tiene el \emph{test} en verde, se procede a refactorizar. Como se
mencionó más arriba, hay que tener especial cuidado en esta fase, ya que
refactorizar significa cambiar el \emph{software} sin alterar su
comportamiento. Si se añade al código una cláusula \textsc{if}, o cualquier otro
tipo de sentencia de control de flujo, ya no se está refactorizando, se está
alterando el comportamiento. Un indicativo de que no se está refactorizando
correctamente es que \emph{tests} escritos en fases anteriores comiencen a
fallar --- Volverse rojos. --- Si esto ocurriese, se debe deshacer lo hecho
hasta que los \emph{tests} fallidos vuelvan a estar en verde.

\section{Desarrollo orientado a comportamiento}
\label{sec:bdd}

El desarrollo orientado a comportamiento -- \acfi{BDD} en inglés -- consiste en
escribir los \emph{test} de forma que expresen el tipo de comportamiento al que
afectan. Siguen una estructura muy clara, llamada ``\textsc{Given, When,
  Then}''. Por ejemplo, un \emph{test} siguiendo esta estructura podría ser
``\emph{\textsc{Dado} un conjunto de datos vacío, \textsc{cuando} se entrena el
  clasificador, \textsc{entonces} se debería producir una excepción de operación
  inválida''}. Como se aprecia, en la primera parte -- \textsc{Given} -- se
establece un contexto para el \emph{test}, es decir, se preparan los datos de
entrada.  En la cláusula \textsc{When}, se llama al código cuyo comportamiento
se quiere probar y por último, la cláusula \textsc{Then} comprueba que el
resultado del código que se está probando coincide con lo que se esperaba.

El objetivo de este tipo de \emph{test} es que sean lo suficientemente
descriptivos para que cualquier persona familiar con el domínio del problema sea
capáz de entenderlo y opinar.

\section{TDD aplicado al Aprendizaje Automático}
\label{sec:tddml}

Una vez se han introducido los conceptos \ac{TDD} y \ac{BDD} se pasa a describir
cómo pueden aplicarse estas filosofías al desarrollo de sistemas de
aprendizaje.

En cada algoritmo de \ac{AA} existe alguna forma de cuantificar la calidad del
resultado. En regresión lineal se ajusta el valor R2, para problemas de
clasificación se utiliza la curva \acfi{ROC} y el área bajo la misma ---
\ac{AUC} --- una matriz de confusión u otro tipo de medidas.

Para empezar a desarrollar un sistema, primero se construye un algoritmo muy
básico e ignorante. La calidad de este algorimo será puramente aleatoria. A
partir de aquí, se puede comenzar a desarrollar otro algoritmo que se comporte
mejor, superando la pura aleatoriedad obtenida anteriormente. Así se comienza un
ciclo itetarivo en el que se intenta superar a las puntuaciones obtenidas en el
paso anterior.

\section{Casos de prueba realizados}
\label{sec:impltests}

En esta sección se introducen los distintos \emph{tests} que se han realizado al
proyecto.

\subsection{\textsc{DataParserSpec}}
\label{sec:dataparser}

En este \emph{test} se comprueban dos situaciones con cláusulas \textsc{Given,
  When, Then}. La primera debe comprobar que cuando no se proporcionan datos
para entrenar el modelo, el algoritmo no debe hacer nada. La segunda comprueba
que cuando sí que se proporcionan datos, se leen correctamente. El código se
lista en \autoref{lst:dataparserspec}.

\begin{listing}[ht]
  \begin{scalacode}
class DataParserSpec extends Specification
  with GWT
  with StandardRegexStepParsers { def is = s2"""
  When no data given, do not launch algorithm   ${testResources.start}
    Given no train data
    Given no test data
    Then should do nothing                      ${testResources.end}
  When data given, launch algorithm             ${dataSet.start}
    Given Train file: es_ancora-converted-train1
    Given Test file: es_ancora-converted-test1
    When Launching
    Then should read Correctly                  ${dataSet.end}
  """

  val dataSetName = readAs(".*: (.*)$").and((s: String) => s)
  val noDataSet = readAs(".*").and((s:String) => "/tmp/aa")

  val testResources =
    Scenario("NoData").
      given(noDataSet).
      given(noDataSet).
      when() {case train :: test :: _ =>

        val trainResource =
           if (getClass.getResource(train) == null) ""
           else getClass.getResource(train).getPath
        val testResource =
           if (getClass.getResource(test) == null) ""
           else getClass.getResource(test).getPath

        val r1 = DataParser.readDataSet(trainResource)
        val r2 =  DataParser.readDataSet(testResource)

        (r1,r2)
      }.
      andThen() {case  _ :: r :: _ => (r._1 must beNone) && (r._2 must beNone) }

  val dataSet = Scenario("DATA").
      given(dataSetName).
      given(dataSetName).
      when(aString) {case _ :: t :: tt :: _ =>
        val r1 = DataParser.
                   readDataSet(
                     getClass.getResource(s"/data/spanish/$t"))
        val r2 = DataParser.
                   readDataSet(
                     getClass.getResource(s"/data/spanish/$tt"))

        (r1,r2)
      }.
      andThen() {case _ :: r :: _ => (r._1 must beSome) && (r._2 must beSome)}

}
  \end{scalacode}
  \caption{\emph{Test} que comprueba si la entrada de datos es correcta o no}
  \label{lst:dataparserspec}
\end{listing}

\subsection{\textsc{DependencyParserCheckBaseLineSpec}}
\label{sec:dpbaseline}

Este \emph{test} se encarga de establecer un punto de partida para los
resultados del modelo. Como se vio en la \autoref{sec:tddred}, este punto de
partida sirve para comprobar que no se están empeorando las predicciones del
modelo conforme vamos desarrollando el sistema. El punto de partida se establece
para dos tipos de medidas, la primera, cuyo \emph{test} se puede ojear en el
Código~\autoref{lst:baseline} comprueba la precisión del modelo fijando unas
cotas mínimas que deben cumplirse para las medidas de evaluación --- pueden
consultarse en la \autoref{sec:eval} --- La segunda comprueba que el número de
características -- \autoref{subsec:featureextraction} -- es siempre el mismo
cuando se usa el conjunto de datos para dearrollo, el código de este \emph{test}
se muestra en \autoref{lst:nfeatures}.
\begin{listing}[ht]
  \begin{scalacode}
class DependencyParserCheckBaselineSpec extends Specification
  with GWT
  with StandardRegexStepParsers {def is = s2"""
  When training the model, set the following baselines  ${featuresBaseline.start}
    Given Train data set: es_ancora-converted-train1
    Given Test data set: es_ancora-converted-test1
    When Genenaring Vocabulary
    Then Dep. Acc should be at least: 70%
    and Root Acc should be at least: 50%
    and Comp. Acc should be at least: 3% ${featuresBaseline.end}
  """

  val aDataSet = readAs(".*: (.*)$").and((s: String) => s)
  val myD = readAs(".*: (\\d+).*$").and((s: String) => s.toDouble)

  val featuresBaseline =
    Scenario("nFeatures").
      given(aDataSet).
      given(aDataSet).
      when(aString){case _ :: test :: train :: _ =>
        val testSentences = DataParser.readDataSet(getClass.getResource(s"/data/spanish/$test").getPath)
        val trainSentences = DataParser.readDataSet(getClass.getResource(s"/data/spanish/$train").getPath)
        val parser = new DependencyParser(trainSentences.get, testSentences.get)
        parser.getAccuracy
      }.
      andThen(myD){case baseline :: r :: _ => r.dependencyAccuracy*100 must be_>(baseline)}.
      andThen(myD){case baseline :: r :: _ => r.rootAccuracy*100 must be_>(baseline)}.
      andThen(myD){case baseline :: r :: _ => r.completeAccuracy*100 must be_>(baseline)}

}    
  \end{scalacode}
  \caption{Código del \emph{test} DependencyParserCheckBaselineSpec}
  \label{lst:baseline}
\end{listing}
\begin{listing}[ht]
  \begin{scalacode}
class DependencyParserCheckNFeaturesSpec extends Specification
  with GWT
  with StandardRegexStepParsers {def is = s2"""
  When training the model, set the following baselines  ${nfeaturesCheck.start}
    Given Train data set: es_ancora-converted-train1
    Given Test data set: es_ancora-converted-test1
    When Genenaring Vocabulary
    Then the number of features must be: 46468          ${nfeaturesCheck.end}
  """

  val aDataSet = readAs(".*: (.*)$").and((s: String) => s)

  val nfeaturesCheck =
    Scenario("nFeatures").
      given(aDataSet).
      given(aDataSet).
      when(aString) { case _ :: test :: train :: _ =>
        val testSentences = DataParser.readDataSet(
           getClass.getResource(s"/data/spanish/$test").getPath)
        val trainSentences = DataParser.readDataSet(
           getClass.getResource(s"/data/spanish/$train").getPath)
        val parser = new DependencyParser(trainSentences.get, testSentences.get)
        parser.nFeatures
      }.
      andThen(anInt) { case baseline :: r :: _ => baseline must_== r }
}
  \end{scalacode}
  \caption{Código del \emph{test} DependencyParserCheckNFeaturesSpec}
  \label{lst:nfeatures}
\end{listing}


%*****************************************
%*****************************************
%*****************************************
%*****************************************
%*****************************************
