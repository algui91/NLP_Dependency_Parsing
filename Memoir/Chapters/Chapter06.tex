%************************************************
\chapter{Test-Driven Machine Learning}
\label{ch:tdml}

En este capítulo se verán los casos de prueba implementados para el proyecto. Se
ha seguido un estilo de tests orientado a problemas de \ac{AA}. En las próximas
secciones se introduce este tipo de práctica.

%************************************************
\section{Test-Driven Development}
\label{sec:tdd}

El \acfi{TDD} se basa en dos principios muy claros:
\begin{itemize}
\item No escribir ninguna línea de código nueva a no ser que se tenga un test
  fallido,
\item Eliminar duplicidades.
\end{itemize}

En esencia, \ac{TDD} es un proceso en el desarrollo de \emph{software} que
permite al desarrollador escribir código que especifica el comportamiento que
poseerá el programa antes de que este sea implementado. La ventaja de este
estilo de desarrollo reside en que a cada paso que se avanza, se obtiene un
\emph{software} completamente funcional, así como el conjunto de
especificaciones que lo definen. En el \ac{TDD} está inherente en cada momento
el desarrollo mediante prueba y error, al igual que en el \ac{AA}.

El proceso al que nos sometemos al adoptar la filosofía \ac{TDD} cambia la forma
en la que se piensa al desarrollar código. Además, el \emph{software} diseñado
como resultado será mucho más modular, permitiendo tener disintos componentes
que se pueden intercambiar en todo el \emph{pipeline}.

Cuando se escribe de antemano la intención del código, antes de implementarlo de
verdad, se aplica una presión al diseño del mismo que evita escribir código del
llamado ``\emph{Por si acaso}''. Con el uso de \ac{TDD}, primero se piensa en un
caso de prueba, se ve que el \emph{software} aún no lo soporta y entonces se
corrige. Si no se es capáz de pensar en un caso de prueba, no se añade código.

\ac{TDD} 

\subsection{El ciclo de \ac{TDD}}
\label{subsec:tddcycle}



\subsubsection{Red}
\label{sec:tddred}

\subsubsection{Green}
\label{sec:tddgreen}

\subsubsection{Refactor}
\label{sec:tddrefactor}


%*****************************************
%*****************************************
%*****************************************
%*****************************************
%*****************************************
