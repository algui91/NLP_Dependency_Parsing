%************************************************
\chapter{Test-Driven Machine Learning}
\label{ch:tdml}
\cite{EL LIBRO DE TDD}
En este capítulo se verán los casos de prueba implementados para el proyecto. Se
ha seguido un estilo de \emph{tests} orientado a problemas de \ac{AA}. En las próximas
secciones se introduce este tipo de práctica.

%************************************************
\section{Test-Driven Development}
\label{sec:tdd}

El \acfi{TDD} se basa en dos principios muy claros:
\begin{itemize}
\item No escribir ninguna línea de código nueva a no ser que se tenga un
  \emph{test} fallido,
\item Eliminar duplicidades.
\end{itemize}

En esencia, \ac{TDD} es un proceso en el desarrollo de \emph{software} que
permite al desarrollador escribir código que especifica el comportamiento que
poseerá el programa antes de que este sea implementado. La ventaja de este
estilo de desarrollo reside en que a cada paso que se avanza, se obtiene un
\emph{software} completamente funcional, así como el conjunto de
especificaciones que lo definen. En el \ac{TDD} está inherente en cada momento
el desarrollo mediante prueba y error, al igual que en el \ac{AA}.

El proceso al que nos sometemos al adoptar la filosofía \ac{TDD} cambia la forma
en la que se piensa al desarrollar código. Además, el \emph{software} diseñado
como resultado será mucho más modular, permitiendo tener disintos componentes
que se pueden intercambiar en todo el \emph{pipeline}.

Cuando se escribe de antemano la intención del código, antes de implementarlo de
verdad, se aplica una presión al diseño del mismo que evita escribir código del
llamado ``\emph{Por si acaso}''. Con el uso de \ac{TDD}, primero se piensa en un
caso de prueba, se ve que el \emph{software} aún no lo soporta y entonces se
corrige. Si no se es capáz de pensar en un caso de prueba, no se añade código.

\ac{TDD} opera a varios niveles. Los \emph{tests} pueden escribirse para
funciones, métodos, clases, programas, servícios webs, redes neuronales y
\emph{pipeline} de \ac{AA} al completo. En todo momento, independientemente del
nivel, los \emph{tests} se escriben desde la perspectiva del cliente. En este
proyecto se ha centrado el tipo de \emph{tests} hacia el \ac{AA}. En este
contexto, los \emph{tests} se escriben para funciones, métodos, clases,
implementaciones matemáticas y todos los algoritmos de aprendizaje.

\subsection{El ciclo de \ac{TDD}}
\label{subsec:tddcycle}

El ciclo para \ac{TDD} consiste en escribir trozos pequeños de funciones que
intenten hacer algo que aún no está implementado. Normalmente se suele
estructura el \emph{test} en tres partes principales. En la primera se preparan
los objetos y datos necesarios. En la segunda se llama al código para el cual se
está escribiendo el \emph{test}. Por último, se valida si el resultado del
código es el que se esperaba. En la primera fase del \emph{test} se escribe el
código necesario para hacer pasar el \emph{test} --- en este momento no es
relevante que el código sea correcto o esté bien diseñado, el único objetivo es
hacer pasar el \emph{test}. --- Una vez se tiene el \emph{test} correcto, se
procede a refactorizar el código. Cabe destacar que en este contexto
\emph{refactorizar} significa cambiar la forma en la que se escribió el código,
pero bajo ningún concepto cambiar cómo se comporta.

Por tanto, el ciclo del \ac{TDD} se puede dividir en tres pasos: \textsc{Rojo},
\textsc{Verde} y \textsc{Refactorizar}.

\subsubsection{Rojo}
\label{sec:tddred}

Primero se crea un \emph{test} que no funciona. Al más alto nivel en cuanto a
\ac{AA} se refiere, podría ser un punto de partida como conseguir un resultado
mejor que una predicción aleatoria.

\subsubsection{Verde}
\label{sec:tddgreen}

\subsubsection{Refactorizar}
\label{sec:tddrefactor}


%*****************************************
%*****************************************
%*****************************************
%*****************************************
%*****************************************
