%************************************************
\chapter{Conclusiones y Vías Futuras}
\label{ch:future}
%************************************************

En este trabajo se ha implementado un parseador de dependencias para el Español,
ya que la gran mayoría de sistemas se basan en Inglés. El parseo se ha llevado a
cabo mediante el entrenamiento de \acfi{SVM}. El lenguaje elegido ha sido
\textsc{Scala}, por presentar su uso varias ventajas en problemas de \acf{AA} y
\emph{Big Data}. Además, se ha seguido un nuevo paradigma de desarrollo usando
\acfi{TDD} aplicado a problemas de \acf{AA}.

\section{Trabajo Futuro}
\label{sec:future}

Como trabajo futuro se podrían implementar distintos tipos de algoritmos y
establecer una comparación entre ellos.

Otra posible mejora puede ser aprovechar aún más las características de
\textsc{Scala}, ya que al momento de escribir el programa, el desarrollador se
estaba familiarizando con el lenguaje. Hay mucho margen para realizar mejoras en
el código. Por ejemplo, se podría hacer todo el código funcional e
inmutable.

Ya que este algoritmo se ha encargado de implementar un único proceso del
\emph{pipeline} -- \autoref{sec:genericpipeline} y \autoref{sec:corenlppipeline}
-- , se podrían proponer más implementaciones del mismo, hasta llegar a realizar
un \emph{software} que implemente un \emph{pipeline} completo para el
castellano.

Por último, en orden de mejorar el tiempo de entrenamiento, sería posible
transladarse a tecnologías de \emph{Big Data} e implementar el proceso usando
\emph{frameworks} como \textsc{Spark}.

%*****************************************
%*****************************************
%*****************************************
%*****************************************
%*****************************************
