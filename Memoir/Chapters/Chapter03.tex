%************************************************
\chapter{Una Introducción a Scala}
\label{ch:scalaintro}
%************************************************

El nombre \textsc{Scala} es una concatenación de dos palabras, \emph{Scalable
  Language}. A continuación se enumeran algunas de las razones por las que se ha
elegido este lenguaje, la lista completa puede encontrarse en
\citeauthor{Dean2015} \cite{Dean2015}.

\section{¿Por qué \textsc{Scala}?}
\label{sec:whyscala}

Las principales características de \textsc{Scala} que lo hacen un buen candidato
para este trabajo son las siguientes:

\begin{description}
\item[Paradigma mixto --- Programación Orientada a Objetos:] \textsc{Scala}
  soporta al completo el paradigma de la orientación a objetos. Además, mejora
  el modelo de objetos proporcionado por \textsc{Java} con la introducción de
  \textsc{Traits}, un modo muy claro de implementar tipos mediante composiciones
  mixtas. Todo es un objeto en \textsc{Scala}, incluso los tipos numéricos.
\item [Paradigma mixto --- Programación Funcional:] De igual modo,
  \textsc{Scala} soporta al completo \acfi{FP}\acused{FP}\graffito{\ac{FP}:
    Programación Funcional}. En los últimos años, la \ac{FP} ha resurgido como
  una de las mejores herramientas para pensar en problemas de concurrencia,
  \emph{Big Data} y en general para escribir código correcto. Este código
  correcto, conciso y potente se logra mediante el uso de valores inmutables,
  funciones de primera clase, funciones sin efectos colaterales, funciones de
  ``orden superior'' y colecciones de funciones.
\item [Sintaxis breve, elegante y flexible:] Expresiones que pueden llegar a
  ser demasiado extensas en \textsc{Java} se hacen concisas en \textsc{Scala}.
\item [Arquitectura Scalable:] \textsc{Scala} permite escribir desde
  \emph{scripts} pequeños, que son interpretados, hasta aplicaciones
  distribuidas de gran envergadura. Hay cuatro mecanismos inherentes al lenguaje
  permitiendo esta escalabilidad: 1) composiciones mixtas mediante
  \textsc{Traits}, 2) miembros de tipo abstracto y genéricos; 3) anidamiento de
  clases y 4) tipos explícitos \textsc{self}.
\end{description}

\section{Patrones de diseño como miembros de primera clase}
\label{sec:patterns}

Otra de las ventajas de \textsc{Scala} es que incorpora ciertos patrones de
diseño en el mismo lenguaje. Por ejemplo, en \nameref{lst:object} se muestra una
implementación del patrón.

\begin{listing}[H]
  \begin{scalacode}
    object Upper {
      def upper(strings: String*) = strings.map(_.toUpperCase())
    }
    println(Upper.upper("Hello", "World!"))
  \end{scalacode}
  \caption{Patrón Singleton en Scala.}
  \label{lst:object}
\end{listing}

En la primera línea, \scalainline/object Upper/ crea el objeto
\emph{singleton}. El patrón \emph{singleton} tiene sentido cuando no es
necesario guardar ningún estado del objeto ni el objeto interactua con el mundo
exterior.

Es posible refactorizar el código anterior, como se muestra en
\nameref{lst:object2}

\begin{listing}[H]
  \begin{scalacode}
    object Upper2 {
      def main(args: Array[String]) = {
        val output = args.map(_.toUpperCase()).mkString(" ")
        println(output)
      }
    }
  \end{scalacode}
  \caption{Refactorizando Upper}
  \label{lst:object2}
\end{listing}

\section{Reglas de Visibilidad}
\label{sec:visibility}

\textsc{Scala} ofrece un amplio abanico de posibilidades en cuanto a la
visibilidad a la que se exponen las clases, objetos y métodos. Este abanico es
mucho más amplio que el ofrecido por \textsc{Java} y se puede consultar en la
\autoref{table:visibility}

\begin{table}[H]
\centering
\caption{Reglas de visibilidad en \textsc{Scala}}
\label{table:visibility}
\begin{tabular}{llp{4cm}}
\rowcolor[HTML]{443627} 
{\color[HTML]{FFFFFF} Nombre} & {\color[HTML]{FFFFFF} Palabra reservada} & {\color[HTML]{FFFFFF} Descripción} \\
public & ninguna, por defecto &  Visibles en cualquier lugar. \\
protected & \scalainline/protected/ &  Visibles al tipo que los define, tipos
                                      derivados y tipos anidados. Solo son
                                      visibles dentro del mismo paquete y sub paquetes.\\
private & \scalainline/private/ &  Solo en los tipos que los definen y tipos
                                  anidados. A nivel de paquete son visibles
                                  únicamente en el mismo que los define.\\
scoped protected & \scalainline/protected[scope]/ &  Visibilidad delimitada por
                                                    el ámbito definido por
                                                    \textsc{scope}. \textsc{Scope}
  puede ser un paquete, tipo o \scalainline/this/ --- misma instancia ---\\
scoped private & \scalainline/private[scope]/ & Similar al anterior, pero los
                                                objetos derivados no pueden
                                                acceder.
\end{tabular}
\end{table}

\section{Big Data}
\label{sec:bigdata}

La principal razón del éxito de \textsc{Scala} en la comunidad del \ac{AA} y
\emph{Big Data} reside en la facilidad que ofrece para escribir programas
concurrentes usando el paradigma \ac{FP}. La diferencia entre aplicaciones de
\emph{Big Data} escritas en \textsc{Java} frente a las escritas en
\textsc{Scala} mediante \ac{FP} es abismal. En \nameref{lst:object2} se vio un
ejemplo de \scalainline/map/, funciones como esta y sus compañeras ---
\scalainline/flatMap, filter, fold/\dots --- han sido siempre herramientas para
trabajar con datos. Independientemente del tamaño de los datos, se aplica la
misma abstracción.

Como ejemplo mostraremos dos versiones del clásico \textsc{Hola Mundo} de
\textsc{Map Reduce}, una en \textsc{Java} -- \autoref{lst:javaWC} -- y otra en
\textsc{Scala} -- \autoref{lst:scalaWC} -- --- Los ejemplos son del libro de
\citeauthor{Dean2015} \cite{Dean2015}. Basta observar el código para notar la
verbosidad de \textsc{Java} frente a \textsc{Scala}.

\begin{listing}[H]
  \begin{javacode}
    class WordCountMapper extends MapReduceBase
    implements Mapper<IntWritable, Text, Text, IntWritable> {

      static final IntWritable one  = new IntWritable(1);
      // Value will be set in a non-thread-safe way!
      static final Text word = new Text;

      @Override
      public void map(IntWritable key, Text valueDocContents,
      OutputCollector<Text, IntWritable> output, Reporter reporter) {
        String[] tokens = valueDocContents.toString.split("\\s+");       
        for (String wordString: tokens) {
          if (wordString.length > 0) {
            word.set(wordString.toLowerCase);
            output.collect(word, one);
          }
        }
      }
    }

    class WordCountReduce extends MapReduceBase
    implements Reducer<Text, IntWritable, Text, IntWritable> {

      public void reduce(Text keyWord, java.util.Iterator<IntWritable> counts,
      OutputCollector<Text, IntWritable> output, Reporter reporter) {
        int totalCount = 0;
        while (counts.hasNext) {                                         
          while (counts.hasNext) {
            totalCount += counts.next.get;
          }
          output.collect(keyWord, new IntWritable(totalCount));
        }
      }
  \end{javacode}
  \caption{\textsc{WordCount} en \textsc{Java}}
  \label{lst:javaWC}
\end{listing}

\begin{listing}[H]
  \begin{scalacode}
    class ScaldingWordCount(args : Args) extends Job(args) {
      TextLine(args("input"))
      .read
      .flatMap('line -> 'word) {
        line: String => line.trim.toLowerCase.split("""\s+""")
      }
      .groupBy('word){ group => group.size('count) }
      .write(Tsv(args("output")))                   
    }
  \end{scalacode}
  \caption{\textsc{WordCount} en \textsc{Scala}}
  \label{lst:scalaWC}
\end{listing}

Podríamos decir que estos son los motivos principales -- aunque no los únicos --
de la elección de este lenguaje de programación en el desarrollo de un parseador
de dependencias para Español.

%*****************************************
%*****************************************
%*****************************************
%*****************************************
%*****************************************
