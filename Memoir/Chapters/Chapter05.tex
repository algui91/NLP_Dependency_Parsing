%************************************************
\chapter{Implementación}
\label{ch:impl}
%************************************************

En este capítulo se detalla todo el proceso llevado a cabo para el desarrollo
del proyecto, desde la planificación hasta la finalización del mismo. La
implementación del algoritmo se ha realizado en \textsc{Scala}, los detalles del
mismo pueden encontrarse en el \autoref{ch:algorithm}. Así mismo, las ventajas
del desarrollo en \textsc{scala} pueden consultarse en el
\autoref{ch:scalaintro}.

\section{Planificación}
\label{sec:planning}

En la \autoref{fig:planning} se muestra un diagrama de \emph{Gantt} con la
planificación ideada para el proyecto

\begin{figure}[ht]
  \definecolor{barblue}{RGB}{153,204,254}
  \definecolor{groupblue}{RGB}{51,102,254}
  \definecolor{linkred}{RGB}{165,0,33}
  \renewcommand\sfdefault{phv}
  \renewcommand\mddefault{mc}
  \renewcommand\bfdefault{bc}
  \setganttlinklabel{s-s}{START-TO-START}
  \setganttlinklabel{f-s}{FINISH-TO-START}
  \setganttlinklabel{f-f}{FINISH-TO-FINISH}
\begin{ganttchart}[
  canvas/.append style={fill=none, draw=black!5, line width=.75pt},
  hgrid style/.style={draw=black!5, line width=.75pt},
  vgrid={*1{draw=black!5, line width=.75pt}},
  today=14,
  today rule/.style={
    draw=black!64,
    dash pattern=on 3.5pt off 4.5pt,
    line width=1.5pt
  },
  today label font=\small\scshape,
  title/.style={draw=none, fill=none},
  title label font=\scshape\footnotesize,
  title label node/.append style={below=7pt},
  include title in canvas=false,
  bar label font=\mdseries\small\color{black!70},
  bar label node/.append style={left=2cm},
  bar/.append style={draw=none, fill=black!63},
  bar incomplete/.append style={fill=barblue},
  bar progress label font=\mdseries\footnotesize\color{black!70},
  group incomplete/.append style={fill=groupblue},
    group left shift=0,
    group right shift=0,
    group height=.5,
    group peaks tip position=0,
    group label node/.append style={left=.6cm},
    group progress label font=\scshape\small,
    link/.style={-latex, line width=1.5pt, linkred},
    link label font=\scriptsize\scshape,
    link label node/.append style={below left=-2pt and 0pt},
  ]{1}{14}
  \gantttitle{Parseo de Dependencias en Español}{14} \\[grid]
  \gantttitle{\tiny Septiembre}{4}
  \gantttitle{\tiny Octubre}{4} 
  \gantttitle{\tiny Noviembre}{4}
  \gantttitle{\tiny Diciembre}{2}\\
  \gantttitle[
    title label node/.append style={below left=7pt and -3pt}
  ]{Semana:\quad1}{1}
  \gantttitlelist{2,...,14}{1} \\
  \ganttgroup[progress=100]{Progreso}{1}{14} \\
  \ganttbar[
    progress=100,
    name=research
  ]{Investigación}{1}{4} \\
  \ganttbar[
    progress=100,
    name=design
  ]{Análisis y Diseño}{5}{5} \\
  \ganttbar[
    progress=100,
    name=impl
  ]{Implementación}{6}{11} \\
  \ganttbar[
    progress=100,
    name=memoir
  ]{Memoria}{12}{14} \\    
  
  \ganttmilestone{M1: Conocer el campo del NLP}{4}{4}  \\
  \ganttmilestone{M2: Finalizar Código}{11}{11} \\
  \ganttmilestone{M3: Finalización TFG}{14}{14}
  
  \ganttlink[link type=f-s]{research}{design}
  \ganttlink[link type=f-s]{design}{impl}
  \ganttlink[link type=f-s]{impl}{memoir}
\end{ganttchart}
\caption{Planificación del proyecto}
\label{fig:planning}
\end{figure}

\section{Análisis y Diseño}
\label{sec:design}

\tikzumlset{fill package=gray!20, fill class=gray!20}
Los paquetes creados se organizan según la \autoref{fig:packages}.
\begin{figure}[ht]

  \makebox[\textwidth][c]{
  \begin{tikzpicture}
    \begin{umlpackage}{Core}
      \umlemptypackage{Functional}
    \end{umlpackage}
    \umlemptypackage[right=1cm of Core,anchor=north]{DataStructures}
    \umlemptypackage[right=0.22cm of DataStructures]{Parser}
    \umlemptypackage[right=0.22cm of Parser]{SVM}
    \umlemptypackage[right=0.22cm of SVM]{Utils}
  \end{tikzpicture}
  }
  \caption{Paquetes del proyecto}
  \label{fig:packages}
\end{figure}

En el paquete \textsc{core.functional} se definen algunas estructuras de teoría
de categorías, actualmente solo hay implementada una mónada -- \emph{monads} --.

En \textsc{DataStructures} se definen las estructuras de datos necesarias para
el desarrollo del proyecto, entre otras, aquí se definen las representaciones de
las frases para \emph{training} y \emph{test} vistas en la
\autoref{fig:classdiag}. En el Código~\ref{lst:ds} se listan algunas de las
estructuras más relevantes.
\begin{listing}[ht]
  \begin{scalacode}
    // Información sobre un nodo
    case class Node(lex: String,
                    position: Int,
                    posTag: String,
                    var dependency: Int = -1,
                    var left: Vector[Node],
                    var right: Vector[Node])

    // Encargada de almacenar las características para la SVM
    final case class Vocabulary(positionVocab: Map[Int, Counter],
                                positionTag: Map[Int, Counter],
                                chLVocab: Map[Int, Counter],
                                chLTag: Map[Int, Counter],
                                chRVocab: Map[Int, Counter],
                                chRTag: Map[Int, Counter])
  \end{scalacode}
  \caption{\footnotesize Estructuras de datos más relevantes del paquete
    \textsc{DataStructures}}
  \label{lst:ds}
\end{listing}

\textsc{Parser} es el paquete principal, contiene la implementación del
algoritmo de parseo de dependencias estadístico de \citeauthor{yamada2003}.

\textsc{SVM} encapsula todo lo relacionado con las \acp{SVM}, desde el adaptador
para los datos hasta la configuración y ajuste de parámetros. Lo más relevante
quizá sean los parámetros usados para la \ac{SVM}, se muestran en el
Código~\ref{lst:svmparams}.
\begin{listing}[ht]
  \begin{scalacode}
    object SVMConfig {
      val param = new svm_parameter

      param.svm_type = svm_parameter.C_SVC
      param.kernel_type = svm_parameter.POLY
      param.degree = 2
      param.gamma = 1.0
      param.coef0 = 1.0
      param.cache_size = 4000
      param.eps = 0.001
      param.C = 1.0
      param.shrinking = 1
    }
  \end{scalacode}
  \caption{Parámetros para la \ac{SVM}}
  \label{lst:svmparams}
\end{listing}
Se aprovecha el Código~\ref{lst:svmparams} para comentar los parámetros usados:
\begin{itemize}
\item \scalainline/param.svm_type = svm_parameter.C_SVC/: especifica que el tipo
  de clasificación va a ser multiclase.
\item \scalainline/param.kernel_type = svm_parameter.POLY/: Como se comentó en
  la \autoref{sec:svmintro} el \emph{kernel} será de tipo polinómico, de grado
  2. El \emph{kernel} se define como $(\gamma\cdot u'\cdot v + coef0)^{degree}$,
  cuyos valores pueden consultarse en el código.
\end{itemize}

\textsc{Utils} define algunas constantes, tipos de datos, métodos de lectura
para los datos de \emph{test} y \emph{training} y encapsula los tres tipos de
acciones que puede realizar el parseador. Las acciones se han codificado según
el Código~\ref{lst:actions}.
\begin{listing}[ht]
  \begin{scalacode}
    object Action {

      sealed trait Action

      case object Left extends Action {
        final def value: Int = 0
      }

      case object Shift extends Action {
        final def value: Int = 1
      }

      case object Right extends Action {
        final def value: Int = 2
      }
    }
  \end{scalacode}
  \caption{\footnotesize Codificación de las acciones \textsc{Desplazar,Izquierda,Derecha}}
  \label{lst:actions}
\end{listing}

El diagrama de clases del proyecto pueden consultarse en la
\autoref{fig:classdiag}

\begin{figure}[ht]
  \makebox[\textwidth][c]{
  \begin{tikzpicture}
    \umlclass{Accuracy}{
      -- rootAcc:Map[Str,Int]\\
      -- depNAcc:Map[Str,Int]\\
      -- depDAcc:Map[Str,Int]\\
      -- completeD:Int\\
      -- completeN:Int
    }{
      + rootAcc:Double\\
      + depAcc:Double\\
      + compAcc:Double
    }
    
    \umlclass[right=3cm of Accuracy]{DepParsing}{
      + trainSnt:Vector[LabeledSentence]\\
      + testSnt:Vector[LabeledSentence]\\
    }{
      + getAcc:Accuracy
    }
    
    \umlnest{DepParsing}{Accuracy}
    
    \umlclass[above=1cm of DepParsing]{LabeledSentence}{
      + t:Tokens
    }{}
    
    \umlassoc{DepParsing}{LabeledSentence}
    
    \umlclass[left=5mm of LabeledSentence]{Tokens}{
      + lex:Vector[Str]\\
      + pos:Vector[Str]\\
      + gold:Vector[Str]\\
      + dep:Vector[Int]
    }{}
    
    \umlclass[above right=3cm and 2mm of LabeledSentence.north,
              type=trait]{TrainSentence}{
      + dep:Vector[Int]
    }{}
    
    \umlclass[above=5mm of LabeledSentence,left=5mm of TrainSentence,
              type=trait]{TestSentence}{
      + words:Vector[Str]\\
      + tags:Vector[Str]\\
      + tree:Vector[Str]
    }{
      + size:Int
    }
    
    \umlVHVinherit{LabeledSentence}{TestSentence}
    \umlVHVinherit{LabeledSentence}{TrainSentence}
    
    \umlclass[x=0,y=0,left=5mm of TestSentence]{Node}{
      +lex:Str\\
      +pos:Int\\
      +posTag:Str\\
      +dep:Int\\
      +left:Vector[Node]\\
      +right:Vector[Node]
    }{}
    
    \umlassoc[mult=0..*,pos=0.5,recursive=-190|190|1.5cm]{Node}{Node}
    \umlassoc[mult2=1,mult1=*]{Node}{TestSentence}
    \umlassoc{LabeledSentence}{Tokens}
    
    \umlclass[below=1.5cm of DepParsing]{SVMAdapter}{}{
      + createNode(f:Vector[Int]):Array[svm\_node]\\
      + trainSVM(p:SVMProblem,pa:svm\_parameter):svm\_model\\
      + predictSVM(m:svm\_model,x:Vector[Int]):Double\\
      -- toNodes(x:Vector[Int]):Array[svm\_node]
    }
    
    \umlassoc{SVMAdapter}{DepParsing}
    
    \umlclass[left=5mm of SVMAdapter]{SVMProblem}{
      + numObs:Int\\
      + labels:Array[Double]
    }{
      + update(n:Int,node:Array[svm\_node])
    }
    
    \umlnest{SVMAdapter}{SVMProblem}
  \end{tikzpicture}
}
\caption{Diagrama de clases completo}
\label{fig:classdiag}
\end{figure}

\section{Implementación}
\label{sec:ch5impl}

En esta sección se mostrará en detalle el código implementado del diseño visto
en la \autoref{sec:design}. Se comenzará con el código de la clase principal,
que implementa el algoritmo de aprendizaje, llamado \textsc{DependencyParser}.

\subsection{\textsc{DependencyParser}}
\label{subsec:depparser}

Para facilitar la lectura de la implementación, se dividirá el código en varias
partes y se comentarán por separado.
\begin{scala2}
class DependencyParser(val trainSentences: Vector[LabeledSentence],
                       val testSentences: Vector[LabeledSentence]) {

  case class Accuracy(private[DependencyParser] val rootAcc: Map[String, Int] = Map.empty,
                      private[DependencyParser] val depNAcc: Map[String, Int] = Map.empty,
                      private[DependencyParser] val depDAcc: Map[String, Int] = Map.empty,
                      private[DependencyParser] val completeD: Int = 0,
                      private[DependencyParser] val completeN: Int = 0){

    def rootAccuracy: Double = rootAcc.values.sum / testSentences.size.toDouble
    def dependencyAccuracy: Double = depNAcc.values.sum / depDAcc.values.sum.toDouble
    def completeAccuracy: Double = completeN / completeD.toDouble
  }
// ...
\end{scala2}
La declaración de la clase principal --- \textsc{DependencyParser} --- indica
que debe recibir en su constructor dos conjuntos de datos, el de \emph{train} y
el de \emph{test}. Así mismo, se declara como clase interna \textsc{Accuracy},
cuya función es calcular las medidas de evaluación -- \autoref{sec:eval}. --

El resto de datos miembro de la clase \textsc{DependencyParser} son
\begin{scala2}
  //...
  /**
    * Train
    */
  // 1.1 - Build Vocabulary
  private[this] val vocabulary = generateVocabulary(trainSentences)
  // 1.2 - Extract Features
  private[this] val features = extractFeatures(sentences2)
  // 1.3 - Train models
  private[this] val models = train(features._1, features._2)

  val nFeatures = vocabulary.nFeatures

  /**
    * Test with unseen data
    */
  private[this] val inferredTree = test(testSentences)
  //...
\end{scala2}
El primer paso del algoritmo es calcular el vocabulario a usar en el proceso de
entrenamiento, mediante el método \textsc{generateVocabulary}. El siguiente paso
consiste en extraer las características -- \autoref{table:features} -- usando el
método \textsc{extractFeatures} -- \autoref{subsec:featureextraction} --. Por
último, solo resta entrenar los modelos mediante el método \textsc{Train} usando
las características extraidas.

Se procede ahora a mostar la implementación de \textsc{generateVocabulary}
\begin{scala2}
  // 1.1 - Build Vocab
  private[this] def generateVocabulary(sentences: Vector[LabeledSentence]): Vocabulary
\end{scala2}
El modo de calcular el vocabulario usa el Algoritmo~\autoref{algorithm:parsing}
propuesto por \citet{yamada2003}. La estimación de la acción a tomar --
\autoref{subsec:parseractions} -- se realiza de forma algorítitmica, sin
involucrar al modelo \ac{SVM}. Es en este método donde se construye el árbol de
dependencias.

Tras construir el vocabulario, se extraen las características con
\begin{scala2}
  // 1.2 - Extract features
  private[this] def extractFeatures(sentences: Seq[LabeledSentence]): (Map[String, Vector[Vector[Int]]], Map[String, DblVector]) = {
\end{scala2}
El método se encarga de generar las características que le serán proporcionadas
a la \ac{SVM} para entrenarse.

Por último, resta la fase de entrenamiento, de la cual se encarga el método
\textsc{Train}. Debido a la importancia de este método, se muestra su código al
completo.
\begin{scala2}
  // 1.3 - Train models
  def train(X: Map[String, Vector[Vector[Int]]], Y: Map[String, DblVector]): Map[String, svm_model] = {
    logger.info("Training models...")

    val nFeatures = vocabulary.nFeatures

    @tailrec
    def train0(XKey: Iterable[String], modelsAcc: Map[String, svm_model]): Map[String, svm_model] =
      (XKey.toSeq: @switch) match {
        case head +: tail =>

          logger.debug(s"\t\tPosTags left: $XKey")
          logger.debug(s"\t\tSize: ${X(head).size}")
          logger.debug(s"\t\t# features: $nFeatures")
          (getClass.getResource(s"${Constants.ModelPath}/svm.$head.model") != null: @switch) match {
            case true =>
              val modelPath = getClass.getResource(s"${Constants.ModelPath}/svm.$head.model").getPath
              logger.info(s"Loaded model: ${modelPath.substring(modelPath.indexOf("svm"))}")
              logger.debug(s"Loaded model: $modelPath")
              // Load Models
              train0(tail, modelsAcc + (head -> svm.svm_load_model(modelPath)))
            case false =>
              val svmProblem = new SVMProblem(Y(head).size, Y(head).toArray)
              // Create each row with its feature values Ex: (Only store the actual values, ignore zeros)
              //   x -> [ ] -> (2,0.1) (3,0.2) (-1,?)
              //        [ ] -> (2,0.1) (3,0.3) (4,-1.2) (-1,?)
              //        ......................................
              X(head).zipWithIndex.foreach {
                case (x, i) =>
                  val nodeCol = createNode(x)
                  svmProblem.update(i, nodeCol)
              }
              val error = svm.svm_check_parameter(svmProblem.problem, SVMConfig.param)
              require(error == null, f"${logger.error(s"Errors in SVM parameters:\n$error")}")

              val m = modelsAcc + (head -> trainSVM(svmProblem, SVMConfig.param))
              svm.svm_save_model(s"src/main/resources/models/svm.$head.model", m(head))

              train0(tail, m)
          }
        case Nil => modelsAcc
      }
    train0(X.keys, Map.empty)
  }
\end{scala2}
El método recibe como parámetro las características calculadas en el paso
anterior. En su interior reside una función recursiva que recorre todas las
características y va generando tantas \acp{SVM} como sea necesario, como se
explicó en la \autoref{subsec:svmgrouping}. En el caso de que ya existan los
modelos de una anterior ejecución, estos son cargados en lugar de calcularse de
nuevo, lo cual ahorra tiempo, ya que entrenar los modelos puede llevar hasta dos
días.

Una vez se tienen los modelos entrenados, es el momento de comprobar la calidad
de las predicciones, esto se logra con el método \textsc{test}.
\begin{scala2}
private[this] def test(sentences: Vector[LabeledSentence]): Vector[Vector[Node]]
\end{scala2}
El método \textsc{test} genera un vector de árboles inferidos por los modelos
entrenados. Una vez terminado, para calcular las medidas de evaluación se debe
llamar al método \textsc{getAccuracy}
\begin{scala2}
def getAccuracy: Accuracy
\end{scala2}
Que devuelve un objeto de tipo \textsc{Accuracy}, encargado de calcular los tres
tipos de medidas de evaluación: \textsc{Dep Acc, Root Acc} y
\textsc{Comp. Rate}.

\subsection{\textsc{Node}}
\label{subsec:node}

La clase \textsc{Node} representa el árbol de dependencias, debido a su
importancia se muestra el código al completo.
\begin{scala2}
/**
  * Node of a tree
  * Can contain any number of children
  * Left and Right represents the children created due to left and right
  * dependencies, respectively.
  *
  * Created by Alejandro Alcalde <contacto@elbauldelprogramador.com>.
  */
case class Node(lex: String,
                position: Int,
                posTag: String,
                var dependency: Int = -1,
                var left: Vector[Node],
                var right: Vector[Node]) {

  def insertRight(child: Node): Unit = {
    child.dependency = position
    right = right :+ child
  }

  def insertLeft(child: Node): Unit = {
    child.dependency = position
    left = left :+ child
  }

  def matchDep(goldSentence: LabeledSentence, depAcc: Map[String, Int], depAccBase: Map[String, Int])
  : (Map[String, Int], Map[String, Int]) = {

    @tailrec
    def matchDep0(acc: Map[String, Int], acc2: Map[String, Int], n: Node)(queue: Seq[Node])
    : (Map[String, Int], Map[String, Int]) = {

      @inline def condition(node: Node): Boolean =
        node.dependency != -1 && !Constants.punctuationTags.contains(goldSentence.tags(node.position))
      @inline def condition2(node: Node): Boolean =
        goldSentence.dep(node.position) == node.dependency

      val w = goldSentence.words(n.position)
      val newAccs = if (condition(n) && condition2(n)) {
        (acc + (w -> (acc.getOrElse(w, 0) + 1)), acc2 + (w -> (acc2.getOrElse(w, 0) + 1)))
      } else if (condition(n)) {
        (acc, acc2 + (w -> (acc2.getOrElse(w, 0) + 1)))
      } else {
        (acc, acc2)
      }

      (queue: @switch) match {
        case head +: tail => matchDep0(newAccs._1, newAccs._2, head)(head.left ++ head.right ++ tail)
        case Nil => (newAccs._1, newAccs._2)
      }
    }

    matchDep0(depAcc, depAccBase, this)(left ++ right)
  }

  /**
    * Check if the tree is parsed correctly completely (Ignoring punctuation tags)
    * against the Gold sentence tags
    *
    * @param goldSentence The sentences corretly annotated
    * @return True if the tree is 100% correctly parsed
    */
  def matchAll(goldSentence: LabeledSentence): Boolean = matchNodes(goldSentence) == goldSentence.words.size

  def matchNodes(goldSentence: LabeledSentence): Int = {
    @inline def condition(n: Node): Boolean =
      (goldSentence.dep(n.position) == n.dependency) || Constants.punctuationTags.contains(goldSentence.tags(n.position))

    @tailrec
    def match0(acc: Int, n: Node)(queue: Seq[Node]): Int = {
      val count = if (condition(n)) acc + 1 else acc

      (queue: @switch) match {
        case head +: tail => match0(count, head)(head.left ++ head.right ++ tail)
        case Nil => count
      }
    }

    match0(0, this)(left ++ right)
  }

  override def toString: String = s"<LEX: $lex, TAG: $posTag, DEP: $dependency, POS: $position, LEFT: $left, RIGHT:  $right>"
}
\end{scala2}
Esta clase recibe como parámetros:
\begin{itemize}
   \item [Lex:] La palabra que almacena.
   \item [posTag:] La categoría morfosintática de la palabra.
   \item [dependency:] Número representando la dependencia de la palabra.
   \item [left:] Hijos del árbol a la izquierda de este nodo.
   \item [right:] Hijos a la derecha de este nodo.
\end{itemize}
En cuanto a sus métodos:
\begin{itemize}
   \item [\textsc{insertLeft/Right}:] Insertan un nuevo nodo a su izquierda o
     derecha.
   \item [\textsc{matchDep}:] Comprueba que las dependencias son correctas para las
     estimaciones dadas por el modelo.
   \item [\textsc{matchAll}:] Comprueba si el árbol ha sido calculado correctamente al
     completo, es decir, ha tenido un acierto del 100\% para esa frase.
\end{itemize}

%*****************************************
%*****************************************
%*****************************************
%*****************************************
%*****************************************
