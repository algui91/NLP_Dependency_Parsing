%%%%%%%%%%%%%%%%%%%%%%%%%%%%%%%%%%%%%%%%%%%%%%%%%%%%%%%%%%%%%%%%%%%%
%% I, the copyright holder of this work, release this work into the
%% public domain. This applies worldwide. In some countries this may
%% not be legally possible; if so: I grant anyone the right to use
%% this work for any purpose, without any conditions, unless such
%% conditions are required by law.
%%%%%%%%%%%%%%%%%%%%%%%%%%%%%%%%%%%%%%%%%%%%%%%%%%%%%%%%%%%%%%%%%%%%

\documentclass{beamer}
\usetheme[faculty=phil]{fibeamer}
\usepackage[utf8]{inputenc}
\usepackage[
  main=english%, %% By using `czech` or `slovak` as the main locale
                %% instead of `english`, you can typeset the
                %% presentation in either Czech or Slovak,
                %% respectively.
%  czech, slovak %% The additional keys allow foreign texts to be
]{babel}        %% typeset as follows:
%%
%%   \begin{otherlanguage}{czech}   ... \end{otherlanguage}
%%   \begin{otherlanguage}{slovak}  ... \end{otherlanguage}
%%
%% These macros specify information about the presentation
\title{Trabajo Fin de Grado} %% that will be typeset on the
\subtitle{Diseño e implementación de un analizador de dependencias para procesamiento de lenguaje natural en Español} %% title page.
\author{Alejandro Alcalde Barros}
%% These additional packages are used within the document:
\usepackage{ragged2e}  % `\justifying` text
\usepackage{booktabs}  % Tables
\usepackage{tabularx}
\usepackage{tikz}      % Diagrams
\usetikzlibrary{calc, shapes, backgrounds,shapes.geometric,arrows,decorations.markings,babel,positioning,calc}
\usepackage{amsmath, amssymb}
\usepackage{url}       % `\url`s
\usepackage{listings}  % Code listings
\frenchspacing

\usepackage{algorithm}% http://ctan.org/pkg/algorithms
\usepackage{algpseudocode}% http://ctan.org/pkg/algorithmicx
\usepackage{subcaption}

\begin{document}
\frame{\maketitle}

\AtBeginSection[]{% Print an outline at the beginning of sections
  \begin{frame}<beamer>
    \frametitle{Índice Sección \thesection}
    \tableofcontents[currentsection]
  \end{frame}}
  \section{Introducción al NLP}
  \label{sec:intro}
  \begin{frame}[label=intro]{Introducción al NLP}
    Ciencia que estudia la computación lingüística.

    Sus aplicaciones:
    \begin{itemize}
    \item Resúmenes.
    \item Traducción automática.
    \item Reconocimiento de voz.
    \item Sistemas de Diálogo Hablado.
    \item Clasificación de documentos.
    \item Análisis de sentimientos.
    \end{itemize}
  \end{frame}

  \definecolor{tick}{HTML}{5dc452}
  \newcommand*{\checktikz}[1][]{\tikz[x=1em, y=1em]\fill[#1] (0,.35) -- (.25,0) --
      (1,.7) -- (.25,.15) -- cycle;}
  \newcommand*{\ccheck}{\checktikz[tick,rounded corners=.5pt, draw=tick,
      thin]} %\checktikz[rounded corners=.5pt, draw=red, ultra thin]

  \section{Motivación}
  \label{sec:motivacion}
  \begin{frame}[label=moti]{Motivación}
    \framesubtitle{\emph{Pipelines} actuales}
    \begin{table}[!b]
      {\carlitoTLF % Use monospaced lining figures
        \begin{tabularx}{\textwidth}{lllllll}
          \textbf{ANNOTATOR} & \textbf{AR} & \textbf{ZH} & \textbf{EN} & \textbf{FR} & \textbf{DE} & \textbf{ES} \\
          \toprule
             Tokenize/Segment & \ccheck & \ccheck & \ccheck & \ccheck &  & \ccheck \\
             Sentence Split & \ccheck & \ccheck & \ccheck & \ccheck & \ccheck & \ccheck \\
             Part of Speech & \ccheck & \ccheck & \ccheck & \ccheck & \ccheck & \ccheck \\
             Lemma &  &  & \ccheck &  &  &  \\
             Named Entities &  & \ccheck & \ccheck &  & \ccheck & \ccheck \\
             Constituency Parsing & \ccheck & \ccheck & \ccheck & \ccheck & \ccheck & \ccheck \\
             Dependency Parsing &  & \ccheck & \ccheck & \ccheck & \ccheck &  \\
             Sentiment Analysis &  &  & \ccheck &  &  &  \\
             Mention Detection &  & \ccheck & \ccheck &  &  &  \\
             Coreference &  & \ccheck & \ccheck &  &  &  \\
             Open IE &  &  & \ccheck &  &  & \\
          \bottomrule
        \end{tabularx}}
    \end{table}
  \end{frame}

  \section{Algoritmo seleccionado}
  \label{sec:algot}

  \begin{frame}[label=alg]{Algoritmo Seleccionado para Español}
    \framesubtitle{Statistical Dependency Analysis With SVMs}
      \scriptsize
  \algdef{SnE}{Init}{EndInit}{\textbf{Initialize:}}
  \begin{algorithmic}[1] % The number tells where the line numbering should start
    \State \textbf{Input Sentence:} $(w_1, p_1),(w_2,p_2),\cdots,(w_n,p_n)$
    \Init
       \State $i\gets 1$
       \State ${\cal T}\gets \{(w_1, p_1),(w_2,p_2),\cdots,(w_n,p_n)\}$
       \State $\text{no\_construction}\gets \text{true}$
    \EndInit
       \While{$|{\cal T}| \geq 1$}
          \If{$i == |{\cal T}|$}
             \If{no\_construction == true}
                \textbf{break}
             \EndIf
             \State $\text{no\_construction}\gets \text{true}$
             \State $i\gets 1$
          \Else
             \State $\mathbf{x}\gets \text{getContextualFeatures(${\cal T}, i$)} $
             \State $y\gets \text{estimateAction(model, $\mathbf{x}$)}$
             \State construction(${\cal T}, i, y$)
             \If{$y == \text{\emph{Left} or \emph{Right}}$}
                $\text{no\_construction}\gets \text{false}$
             \EndIf
          \EndIf
       \EndWhile
    \end{algorithmic}
  \end{frame}

  \subsection{SVMs}
  \label{subsec:SVM}
  \begin{frame}[label=svm]{SVMs}
    \framesubtitle{Qué es una SVM}
    \begin{figure}[ht]
      \centering
 \tikzset{
   common/.style={
     circle,minimum width=.5ex,draw%draw=white
   },
   supportCommon/.style={
     solid,circle,minimum width=.7ex,thick,draw%draw=white
   },
   txtCommon/.style={text=black},%text=white
   eq/.style={txtCommon,above,right},
   margin/.style={txtCommon,sloped,midway,above},
    lNode/.style={common, fill=none},
    rNode/.style={common, thick, fill},%fill=white},
    supportR/.style={supportCommon, fill},%fill=white},
    supportL/.style={supportCommon, fill=none},
    outerCircle/.style={supportCommon,minimum width=2.8ex,fill=none,draw},
  }
  \begin{tikzpicture}[
        scale=1.8,
        important line/.style={thick}, dashed line/.style={dashed, thin},
        every node/.style={color=black},
    ]   
    \draw[dashed line, yshift=.7cm]
       (.2,.2) coordinate (sls) -- (2.5,2.5) coordinate (sle)
       node[outerCircle] (name) at (2,2){}
       node[supportL] (name) at (2,2){}
       node[outerCircle] (name) at (1.5,1.5){}
       node[supportL] (name) at (1.5,1.5){}
       node [eq] {$w\cdot x + b > 1$};
    
    \draw[important line]
       (.7,.7) coordinate (lines) -- (3,3) coordinate (linee)
       node [eq] {$w\cdot x + b = 0$};
        
    \draw[dashed line, xshift=.7cm]
       (.2,.2) coordinate (ils) -- (2.5,2.5) coordinate (ile)
       node[outerCircle] (name) at (1.8,1.8){}
       node[supportR] (name) at (1.8,1.8){}
       node [eq] {$w\cdot x + b < -1$};

    \draw[very thick,<->] ($(sls)+(.2,.2)$) -- ($(ils)+(.2,.2)$) 
       node[margin] {Margen};        
        
    \foreach \Point in {(.9,2.4), (1.3,2.5), (1,2.1), (2,3), (1,2.9)}{
      \draw \Point node[lNode]{};
    }        
        
    \foreach \Point in {(2.9,1.4), (2.3,.5), (3.3,.1), (2,0.9), (2.5,1)}{
      \draw \Point node[rNode]{};
    }        
  \end{tikzpicture}
  \caption{Ejemplo SVM}
\end{figure}
  \end{frame}

\tikzset
  {common/.style =
    {rectangle, rounded corners, minimum width=1cm, minimum height=1cm,
     text centered, text opacity=1, align=center
    }
  ,notarget/.style = {common, draw=none, opacity=0, text opacity=0.5}
  ,target/.style   = {common, very thick, draw}
  ,blank/.style    = {common, draw=none}
  }

  \subsection{Acciones}
  \label{subsec:acciones}
  \begin{frame}[label=acciones]{Acciones}
    \framesubtitle{Desplazar}
    \begin{figure}[ht]
  \begin{subfigure}[b]{0.43\textwidth}
    \begin{tikzpicture}[node distance=1.1cm]
      \node (n1) [notarget] {I\\\textsc{prp}};
      \node (n2) [target, right of=n1] {saw\\\textsc{vbd}};
      \node (n3) [target, right of=n2] {a\\\textsc{dt}};
      \node (n4) [notarget, right of=n3] {girl\\\textsc{nn}};
      \draw [thick,->] (n4) -- ++(1cm,0) node[sloped,above,midway] {\scriptsize\emph{Shift}};
    \end{tikzpicture}
    \caption{}
  \end{subfigure}
  \qquad
  \begin{subfigure}[b]{0.3\textwidth}
    \begin{tikzpicture}[node distance=1.1cm]
      \node (n1) [notarget] {I\\\textsc{prp}};
      \node (n2) [notarget, right of=n1] {saw\\\textsc{vbd}};
      \node (n3) [target, right of=n2] {a\\\textsc{dt}};
      \node (n4) [target, right of=n3] {girl\\\textsc{nn}};
    \end{tikzpicture}
    \caption{}
  \end{subfigure}
  \caption{Ejemplo de la acción \textsc{Desplazar}. (a) muestra el
    estado antes de aplicar la acción. (b) el resultado tras
    aplicarla}
  \label{fig:shiftaction}
\end{figure}
\end{frame}

\begin{frame}[label=right]{Acciones}
  \framesubtitle{Derecha}
  \begin{figure}[ht]
  \begin{tabular}{p{0.5\textwidth}p{0.5\textwidth}}
    \begin{tikzpicture}[node distance=.5mm,baseline=(n3)]
      \node (n1) [notarget] {I\\\textsc{prp}};
      \node (n2) [notarget, right=of n1] {saw\\\textsc{vbd}};
      \node (n3) [target, right=of n2] {a\\\textsc{dt}};
      \node (n4) [target, right=of n3] {girl\\\textsc{nn}};
      \draw [thick,->] ($(n4.east) + (5mm,0)$) -- ++(.5cm,0) node[above,midway] {\scriptsize\emph{Right}};
    \end{tikzpicture}
  &
    \begin{tikzpicture}[node distance=.5mm,baseline=(n3)]
      \node (n1) [notarget] {I\\\textsc{prp}};
      \node (n2) [notarget, right=of n1] {saw\\\textsc{vbd}};
      \node (non)[blank,right=of n2]{};
      \node (n3) [target, right=of non] {girl\\\textsc{nn}};
      \node (n4) [target, below=5mm of n3,anchor=north] {a\\\textsc{dt}};
      \draw [thick,->] (n4) -- (n3);
    \end{tikzpicture}
  \end{tabular}
  \caption{Ejemplo de la acción \textsc{Derecha}. (a) Estado antes de la
    acción. (b) Estado tras aplicar la acción}
  \label{fig:rightaction}
\end{figure}
\end{frame}

\begin{frame}[label=left]{Acciones}
  \framesubtitle{Izquierda}
  \begin{figure}[ht]
  \centering
  \begin{tabular}{p{0.5\textwidth}p{0.5\textwidth}}
    \begin{tikzpicture}[node distance=.5mm,baseline=(n3)]
      \node (n1) [notarget] {I\\\textsc{prp}};
      \node (n2) [target, right=of n1] {saw\\\textsc{vbd}};
      \node (n3) [target, right=of n2] {girl\\\textsc{nn}};
      \node (n4) [notarget, below=5mm of n3] {a\\\textsc{dt}};
      \draw [thick,->] (n4) -- (n3);
      \draw [thick,->] ($(n3.east) + (1cm,0)$) -- ++(.5cm,0) node[above,midway] {\scriptsize\emph{Left}};
    \end{tikzpicture}
 &
    \begin{tikzpicture}[node distance=.5mm,baseline=(n1)]
      \node (n1) [notarget] {I\\\textsc{prp}};
      \node (n2) [target, right=of n1] {saw\\\textsc{vbd}};
      \node (n4) [target, below=5mm of n2] {girl\\\textsc{nn}};
      \node (n3) [notarget, below=5mm of n4] {a\\\textsc{dt}};
      \draw [thick,->] (n4) -- (n2);
      \draw [thick,->] (n3) -- (n4);
    \end{tikzpicture}
  \end{tabular}
  \caption{Ejemplo de la acción \textsc{Left}. (a) Estado antes de la
    acción. (b) Estado tras aplicar la acción}
  \label{fig:leftaction}
\end{figure}
\end{frame}

\subsection{Ejemplo}
\label{subsec:example}
\begin{frame}[label=example]{Ejemplo}
  \framesubtitle{``\emph{Sobre la oferta de IBM}''}
  \begin{figure}[ht]
  \footnotesize
  \setlength{\extrarowheight}{16pt}
  \begin{tabular}{p{1\textwidth}}
    \begin{tikzpicture}[node distance=1mm,baseline=(sobre)]
      \node (sobre) [target] {Sobre\\\textsc{adp}};
      \node (la) [target, right=of sobre] {la\\\textsc{det}};
      \node (oferta) [notarget, right=of la] {oferta\\\textsc{noun}};
      \node (de) [notarget, right=of oferta] {de\\\textsc{adp}};
      \node (inter) [notarget, right=of de] {IBM\\\textsc{noun}};
      \node (b1) [blank,right=of inter]{};
      \draw [thick,->] (b1.west) -- ++(.5cm,0) node[above,midway]{\tiny\textsc{Shift}};
    \end{tikzpicture}
\\
   \begin{tikzpicture}[node distance=1mm,baseline=(sobre)]
     \node (sobre) [notarget] {Sobre\\\textsc{adp}};
     \node (la) [target, right=of sobre] {la\\\textsc{det}};
     \node (oferta) [target, right=of la] {oferta\\\textsc{noun}};
     \node (de) [notarget, right=of oferta] {de\\\textsc{adp}};
     \node (inter) [notarget, right=of de] {IBM\\\textsc{noun}};
     \node (b1) [blank,right=of inter]{};
   \end{tikzpicture}

  \end{tabular}
\end{figure}
\end{frame}

\begin{frame}[label=example]{Ejemplo}
  \framesubtitle{``\emph{Sobre la oferta de IBM}''}
  \begin{figure}[ht]
    \footnotesize
    \setlength{\extrarowheight}{16pt}
    \begin{tabular}{p{1\textwidth}}
    \begin{tikzpicture}[node distance=1mm,baseline=(sobre)]
      \node (sobre) [notarget] {Sobre\\\textsc{adp}};
      \node (la) [target, right=of sobre] {la\\\textsc{det}};
      \node (oferta) [target, right=of la] {oferta\\\textsc{noun}};
      \node (de) [notarget, right=of oferta] {de\\\textsc{adp}};
      \node (inter) [notarget, right=of de] {IBM\\\textsc{noun}};
      \node (b1) [blank,right=of inter]{};
      \draw [thick,->] (b1.west) -- ++(.5cm,0) node[above,midway]{\tiny\textsc{Right}};
    \end{tikzpicture}
\\
   \begin{tikzpicture}[node distance=1mm,baseline=(sobre)]
     \node (sobre) [target] {Sobre\\\textsc{adp}};
     \node (b1) [blank,right=of sobre] {};
     \node (la) [notarget,below=.3cm of oferta] {la\\\textsc{det}};
     \node (oferta) [target, right=of b1] {oferta\\\textsc{noun}};
     \node (de) [notarget, right=of oferta] {de\\\textsc{adp}};
     \node (inter) [notarget, right=of de] {IBM\\\textsc{noun}};
     \node (b1) [blank,right=of inter]{};
     \draw [thick,->] (la) -- (oferta);
   \end{tikzpicture}
    \end{tabular}
  \end{figure}
\end{frame}

\begin{frame}[label=example]{Ejemplo}
  \framesubtitle{``\emph{Sobre la oferta de IBM}''}
  \begin{figure}[ht]
    \footnotesize
    \setlength{\extrarowheight}{5pt}
    \begin{tabular}{p{1\textwidth}}
    \begin{tikzpicture}[node distance=1mm,baseline=(sobre)]
      \node (sobre) [target] {Sobre\\\textsc{adp}};
      \node (b1) [blank,right=of sobre] {};
      \node (la) [notarget,below=.3cm of oferta] {la\\\textsc{det}};
      \node (oferta) [target, right=of b1] {oferta\\\textsc{noun}};
      \node (de) [notarget, right=of oferta] {de\\\textsc{adp}};
      \node (inter) [notarget, right=of de] {IBM\\\textsc{noun}};
      \node (b1) [blank,right=of inter]{};

      \draw [thick,->] (la) -- (oferta);
      \draw [thick,->] (b1.west) -- ++(.5cm,0) node[above,midway]{\tiny\textsc{Right}};
    \end{tikzpicture}
\\
   \begin{tikzpicture}[node distance=1mm,baseline=(b2)]
     \node (b2) [blank] {};
     \node (b1) [blank,right=of b2] {};
     \node (la) [notarget,below=.3cm of oferta] {la\\\textsc{det}};
     \node (oferta) [target, right=of b1] {oferta\\\textsc{noun}};
     \node (sobre) [notarget,below=.3cm of la] {Sobre\\\textsc{adp}};
     \node (de) [target, right=of oferta] {de\\\textsc{adp}};
     \node (inter) [notarget, right=of de] {IBM\\\textsc{noun}};
     \node (b1) [blank,right=of inter]{};

     \draw [thick,->] (la) -- (oferta);
     \draw [thick,->] (sobre) -- (la);
   \end{tikzpicture}
    \end{tabular}
  \end{figure}
\end{frame}

\begin{frame}[label=example]{Ejemplo}
  \framesubtitle{``\emph{Sobre la oferta de IBM}''}
  \begin{figure}[ht]
    \footnotesize
    \setlength{\extrarowheight}{5pt}
    \begin{tabular}{p{1\textwidth}}
    \begin{tikzpicture}[node distance=1mm,baseline=(b2)]
      \node (b2) [blank] {};
      \node (b1) [blank,right=of b2] {};
      \node (la) [notarget,below=.3cm of oferta] {la\\\textsc{det}};
      \node (oferta) [target, right=of b1] {oferta\\\textsc{noun}};
      \node (sobre) [notarget,below=.3cm of la] {Sobre\\\textsc{adp}};
      \node (de) [target, right=of oferta] {de\\\textsc{adp}};
      \node (inter) [notarget, right=of de] {IBM\\\textsc{noun}};
      \node (b1) [blank,right=of inter]{};

      \draw [thick,->] (la) -- (oferta);
      \draw [thick,->] (sobre) -- (la);

      \draw [thick,->] (b1.west) -- ++(.5cm,0) node[above,midway]{\tiny\textsc{Shift}};
    \end{tikzpicture}
\\
    \begin{tikzpicture}[node distance=1mm,baseline=(b2)]
      \node (b2) [blank] {};
      \node (b1) [blank,right=of b2] {};
      \node (la) [notarget,below=.3cm of oferta] {la\\\textsc{det}};
      \node (oferta) [notarget, right=of b1] {oferta\\\textsc{noun}};
      \node (sobre) [notarget,below=.3cm of la] {Sobre\\\textsc{adp}};
      \node (de) [target, right=of oferta] {de\\\textsc{adp}};
      \node (inter) [target, right=of de] {IBM\\\textsc{noun}};
      \node (b1) [blank,right=of inter]{};

      \draw [thick,->] (la) -- (oferta);
      \draw [thick,->] (sobre) -- (la);
    \end{tikzpicture}
    \end{tabular}
  \end{figure}
\end{frame}

\begin{frame}[label=example]{Ejemplo}
  \framesubtitle{``\emph{Sobre la oferta de IBM}''}
  \begin{figure}[ht]
    \footnotesize
    \setlength{\extrarowheight}{1pt}
    \begin{tabular}{p{1\textwidth}}
    \begin{tikzpicture}[node distance=1mm,baseline=(b2)]
      \node (b2) [blank] {};
      \node (b1) [blank,right=of b2] {};
      \node (la) [notarget,below=.3cm of oferta] {la\\\textsc{det}};
      \node (oferta) [notarget, right=of b1] {oferta\\\textsc{noun}};
      \node (sobre) [notarget,below=.3cm of la] {Sobre\\\textsc{adp}};
      \node (de) [target, right=of oferta] {de\\\textsc{adp}};
      \node (inter) [target, right=of de] {IBM\\\textsc{noun}};
      \node (b1) [blank,right=of inter]{};

      \draw [thick,->] (la) -- (oferta);
      \draw [thick,->] (sobre) -- (la);

      \draw [thick,->] (b1.west) -- ++(.5cm,0) node[above,midway]{\tiny\textsc{Right}};
    \end{tikzpicture}
\\
       \begin{tikzpicture}[node distance=1mm,baseline=(b2)]
      \node (b2) [blank] {};
      \node (b1) [blank,right=of b2] {};
      \node (la) [notarget,below=.3cm of oferta] {la\\\textsc{det}};
      \node (oferta) [target, right=of b1] {oferta\\\textsc{noun}};
      \node (sobre) [notarget,below=.3cm of la] {Sobre\\\textsc{adp}};
      \node (de) [notarget, below=.3cm of inter] {de\\\textsc{adp}};
      \node (b3) [blank,right=of oferta] {};
      \node (inter) [target, right=of b3] {IBM\\\textsc{noun}};

      \draw [thick,->] (la) -- (oferta);
      \draw [thick,->] (sobre) -- (la);
      \draw [thick,->] (de) -- (inter);
    \end{tikzpicture}
    \end{tabular}
  \end{figure}
\end{frame}
\end{document}