\documentclass[utf8,draft]{beamer}
%
% Choose how your presentation looks.
%
% For more themes, color themes and font themes, see:
% http://deic.uab.es/~iblanes/beamer_gallery/index_by_theme.html
%
\overfullrule=5pt
\mode<presentation>
{
  \usetheme{Darmstadt}      % or try Darmstadt, Madrid, Warsaw, ...
  \usecolortheme{beaver} % or try albatross, beaver, crane, ...
  \usefonttheme{default}  % or try serif, structurebold, ...
  \setbeamertemplate{navigation symbols}{}
  \setbeamertemplate{caption}[numbered]
  \setbeamertemplate{footline}[page number]
} 

\institute{ETSIIT}
\date{\today}

\usepackage{fontspec}
\usepackage[EU1]{fontenc}

\usepackage[spanish]{babel}

\subtitle{Trabajo Fin de Grado} %% that will be typeset on the
\title{DISEÑO E IMPLEMENTACIÓN DE UN ANALIZADOR DE DEPENDENCIAS PARA PROCESAMIENTO DE LENGUAJE NATURAL EN ESPAÑOL} %% title page.
\author{Alejandro Alcalde Barros}
%% These additional packages are used within the document:

\usepackage{ragged2e}  % `\justifying` text
\usepackage{booktabs}  % Tables
\usepackage{tabularx}
\usepackage{tikz}      % Diagrams
\usetikzlibrary{calc, shapes, backgrounds,shapes.geometric,arrows,decorations.markings,babel,positioning,calc}
\usepackage{amsmath, amssymb}
\usepackage{url}       % `\url`s
\usepackage{listings}  % Code listings
\frenchspacing

\usepackage{algorithm}% http://ctan.org/pkg/algorithms
\usepackage{algpseudocode}% http://ctan.org/pkg/algorithmicx
\usepackage{subcaption}

\usepackage[outputdir=metafiles]{minted}
\usemintedstyle{tango}
\usepackage{pgfgantt}

\newminted{scala}{
   fontsize=\footnotesize,
   autogobble=true,
   frame=lines,
%   framesep=2mm,
   mathescape=true,
   breaklines,
   breakautoindent,
   mathescape
}
\newminted{java}{
   fontsize=\footnotesize,
   autogobble=true,
   frame=lines,
%   framesep=2mm,
   mathescape=true,
   breaklines,
   breakautoindent,
   mathescape
}

\begin{document}
\frame{\maketitle}

\AtBeginSection[]{% Print an outline at the beginning of sections
  \begin{frame}<beamer>
    \frametitle{Índice Sección \thesection}
    \tableofcontents[currentsection]
  \end{frame}}
  \section{NLP}
  
  \begin{frame}{Introducción al NLP}
    Ciencia que estudia la computación lingüística.
    Sus aplicaciones:
    \begin{itemize}
    \item Resúmenes.
    \item Traducción automática.
    \item Reconocimiento de voz.
    \item Sistemas de Diálogo Hablado.
    \item Clasificación de documentos.
    \item Análisis de sentimientos.
    \end{itemize}
  \end{frame}

  \subsection*{Niveles de Análisis}
  
  \begin{frame}{Niveles de Análisis}
    \begin{itemize}
    \item Documento.
    \item Sentencia.
    \item Entidad.
    \end{itemize}
  \end{frame}

  \definecolor{tick}{HTML}{5dc452}
  \newcommand*{\checktikz}[1][]{\tikz[x=1em, y=1em]\fill[#1] (0,.35) -- (.25,0) --
      (1,.7) -- (.25,.15) -- cycle;}
  \newcommand*{\ccheck}{\checktikz[tick,rounded corners=.5pt, draw=tick,
      thin]} %\checktikz[rounded corners=.5pt, draw=red, ultra thin]

  \subsection*{Motivación}
  
  \begin{frame}[label=moti]{Motivación}
    \framesubtitle{\emph{Pipelines} actuales}
    \begin{table}[!b]
        \begin{tabularx}{\textwidth}{lllllll}
          \textbf{ANNOTATOR} & \textbf{AR} & \textbf{ZH} & \textbf{EN} & \textbf{FR} & \textbf{DE} & \textbf{ES} \\
          \toprule
             Tokenize/Segment & \ccheck & \ccheck & \ccheck & \ccheck &  & \ccheck \\
             Sentence Split & \ccheck & \ccheck & \ccheck & \ccheck & \ccheck & \ccheck \\
             Part of Speech & \ccheck & \ccheck & \ccheck & \ccheck & \ccheck & \ccheck \\
             Lemma &  &  & \ccheck &  &  &  \\
             Named Entities &  & \ccheck & \ccheck &  & \ccheck & \ccheck \\
             Constituency Parsing & \ccheck & \ccheck & \ccheck & \ccheck & \ccheck & \ccheck \\
             Dependency Parsing &  & \ccheck & \ccheck & \ccheck & \ccheck &  \\
             Sentiment Analysis &  &  & \ccheck &  &  &  \\
             Mention Detection &  & \ccheck & \ccheck &  &  &  \\
             Coreference &  & \ccheck & \ccheck &  &  &  \\
             Open IE &  &  & \ccheck &  &  & \\
          \bottomrule
        \end{tabularx}
    \end{table}
  \end{frame}

\section{Objetivos}
\begin{frame}{Objetivos}
  \begin{itemize}
  \item Revisión bibliográfica.
  \item Elección de un parseador y diseño para Scala.
  \item Implementación y TDD.
  \item Evaluación y comparación de resultados.
  \end{itemize}
\end{frame}

  \section{Algoritmo seleccionado}
  \begin{frame}[label=alg]{Algoritmo Seleccionado para Español}
    \framesubtitle{Statistical Dependency Analysis With SVMs}
      \scriptsize
  \algdef{SnE}{Init}{EndInit}{\textbf{Initialize:}}
  \begin{algorithmic}[1] % The number tells where the line numbering should start
    \State \textbf{Input Sentence:} $(w_1, p_1),(w_2,p_2),\cdots,(w_n,p_n)$
    \Init
       \State $i\gets 1$
       \State ${\cal T}\gets \{(w_1, p_1),(w_2,p_2),\cdots,(w_n,p_n)\}$
       \State $\text{no\_construction}\gets \text{true}$
    \EndInit
       \While{$|{\cal T}| \geq 1$}
          \If{$i == |{\cal T}|$}
             \If{no\_construction == true}
                \textbf{break}
             \EndIf
             \State $\text{no\_construction}\gets \text{true}$
             \State $i\gets 1$
          \Else
             \State $\mathbf{x}\gets \text{getContextualFeatures(${\cal T}, i$)} $
             \State $y\gets \text{estimateAction(model, $\mathbf{x}$)}$
             \State construction(${\cal T}, i, y$)
             \If{$y == \text{\emph{Left} or \emph{Right}}$}
                $\text{no\_construction}\gets \text{false}$
             \EndIf
          \EndIf
       \EndWhile
    \end{algorithmic}
  \end{frame}

  \subsection*{SVMs}
  \begin{frame}[label=svm]{SVMs}
    \framesubtitle{Qué es una SVM}
    \begin{figure}[ht]
      \centering
 \tikzset{
   common/.style={
     circle,minimum width=.5ex,draw
   },
   supportCommon/.style={
     solid,circle,minimum width=.7ex,thick,draw
   },
   txtCommon/.style={text=black},
   eq/.style={txtCommon,above,right},
   margin/.style={txtCommon,sloped,midway,above},
   lNode/.style={common, fill=none},
   rNode/.style={common, thick, fill},
   supportR/.style={supportCommon, fill},
   supportL/.style={supportCommon, fill=none},
   outerCircle/.style={supportCommon,minimum width=2.8ex,fill=none,draw},
}
 % % For Dark frames
 % \tikzset{
 %   common/.style={
 %     circle,minimum width=.5ex,draw=white
 %   },
 %   supportCommon/.style={
 %     solid,circle,minimum width=.7ex,thick,draw=white
 %   },
 %   txtCommon/.style={text=white},
 %   eq/.style={txtCommon,above,right},
 %   margin/.style={txtCommon,sloped,midway,above},
 %   lNode/.style={common, fill=none},
 %   rNode/.style={common, thick, fill=white},
 %   supportR/.style={supportCommon, fill=white},
 %   supportL/.style={supportCommon, fill=none},
 %   outerCircle/.style={supportCommon,minimum width=2.8ex,fill=none,draw}
 % }
  \begin{tikzpicture}[
        scale=2,
        important line/.style={thick}, dashed line/.style={dashed, thin},
        every node/.style={color=black},
    ]   
    \draw[dashed line, yshift=.7cm]
       (.2,.2) coordinate (sls) -- (2.5,2.5) coordinate (sle)
       node[outerCircle] (name) at (2,2){}
       node[supportL] (name) at (2,2){}
       node[outerCircle] (name) at (1.5,1.5){}
       node[supportL] (name) at (1.5,1.5){}
       node [eq] {$w\cdot x + b > 1$};
    
    \draw[important line]
       (.7,.7) coordinate (lines) -- (3,3) coordinate (linee)
       node [eq] {$w\cdot x + b = 0$};
        
    \draw[dashed line, xshift=.7cm]
       (.2,.2) coordinate (ils) -- (2.5,2.5) coordinate (ile)
       node[outerCircle] (name) at (1.8,1.8){}
       node[supportR] (name) at (1.8,1.8){}
       node [eq] {$w\cdot x + b < -1$};

    \draw[very thick,<->] ($(sls)+(.2,.2)$) -- ($(ils)+(.2,.2)$) 
       node[margin] {Margen};        
        
    \foreach \Point in {(.9,2.4), (1.3,2.5), (1,2.1), (2,3), (1,2.9)}{
      \draw \Point node[lNode]{};
    }        
        
    \foreach \Point in {(2.9,1.4), (2.3,.5), (3.3,.1), (2,0.9), (2.5,1)}{
      \draw \Point node[rNode]{};
    }        
  \end{tikzpicture}
  \caption{Ejemplo SVM}
\end{figure}
  \end{frame}

  \begin{frame}{SVMs}
    \framesubtitle{Ventajas de uso}
    \begin{itemize}
    \item Gran poder de generalización.
    \item Con el \emph{Kernel Trick} se combinan características.
    \end{itemize}
  \end{frame}
  
\tikzset
  {common/.style =
    {rectangle, rounded corners, minimum width=1cm, minimum height=1cm,
     text centered, text opacity=1, align=center
    }
  ,notarget/.style = {common, draw=none, opacity=0, text opacity=0.5}
  ,target/.style   = {common, very thick, draw}
  ,blank/.style    = {common, draw=none}
  }

  \subsection*{Acciones}
  
  \begin{frame}[label=acciones]{Acciones}
    \framesubtitle{Desplazar}
    \begin{figure}[ht]
      \begin{columns}[onlytextwidth]
        \begin{column}{.5\textwidth}
          \begin{subfigure}{.5\textwidth}
          \begin{tikzpicture}[node distance=1.1cm]
            \node (n1) [notarget] {I\\\textsc{prp}};
            \node (n2) [target, right of=n1] {saw\\\textsc{vbd}};
            \node (n3) [target, right of=n2] {a\\\textsc{dt}};
            \node (n4) [notarget, right of=n3] {girl\\\textsc{nn}};
            \draw [thick,->] (n4) -- ++(1cm,0) node[sloped,above,midway] {\scriptsize\emph{Shift}};
          \end{tikzpicture}
          \caption{}
          \end{subfigure}
        \end{column}
        \begin{column}{.5\textwidth}
          \begin{subfigure}{.5\textwidth}
          \begin{tikzpicture}[node distance=1.1cm]
            \node (n1) [notarget] {I\\\textsc{prp}};
            \node (n2) [notarget, right of=n1] {saw\\\textsc{vbd}};
            \node (n3) [target, right of=n2] {a\\\textsc{dt}};
            \node (n4) [target, right of=n3] {girl\\\textsc{nn}};
          \end{tikzpicture}
          \caption{}
          \end{subfigure}
        \end{column}
      \end{columns}
      \caption{\textsc{Desplazar}. (a) Antes. (b) Después}
    \end{figure}
\end{frame}

\begin{frame}[label=right]{Acciones}
  \framesubtitle{Derecha}

  \begin{figure}[ht]
    \begin{columns}[onlytextwidth]
      \begin{column}{.5\textwidth}
        \begin{subfigure}{.5\textwidth}
          \begin{tikzpicture}[node distance=.5mm,baseline=(n3)]
            \node (n1) [notarget] {I\\\textsc{prp}};
            \node (n2) [notarget, right=of n1] {saw\\\textsc{vbd}};
            \node (n3) [target, right=of n2] {a\\\textsc{dt}};
            \node (n4) [target, right=of n3] {girl\\\textsc{nn}};
            \draw [thick,->] ($(n4.east) + (5mm,0)$) -- ++(.5cm,0) node[above,midway] {\scriptsize\emph{Right}};
          \end{tikzpicture}
          \caption{}
        \end{subfigure}
      \end{column}
      \begin{column}{.5\textwidth}
        \begin{subfigure}{.5\textwidth}
          \begin{tikzpicture}[node distance=.5mm,baseline=(n3)]
            \node (n1) [notarget] {I\\\textsc{prp}};
            \node (n2) [notarget, right=of n1] {saw\\\textsc{vbd}};
            \node (non)[blank,right=of n2]{};
            \node (n3) [target, right=of non] {girl\\\textsc{nn}};
            \node (n4) [target, below=5mm of n3,anchor=north] {a\\\textsc{dt}};
            \draw [thick,->] (n4) -- (n3);
          \end{tikzpicture}
          \caption{}
        \end{subfigure}
      \end{column}
    \end{columns}
    \caption{\textsc{Derecha}. (a) Antes. (b) Después.}
  \end{figure}
\end{frame}

\begin{frame}{Acciones}
  \framesubtitle{Izquierda}
  \begin{figure}[ht]
    \begin{columns}[onlytextwidth]
      \begin{column}{.5\textwidth}
        \begin{subfigure}{.5\textwidth}
          \begin{tikzpicture}[node distance=.5mm,baseline=(n3)]
            \node (n1) [notarget] {I\\\textsc{prp}};
            \node (n2) [target, right=of n1] {saw\\\textsc{vbd}};
            \node (n3) [target, right=of n2] {girl\\\textsc{nn}};
            \node (n4) [notarget, below=5mm of n3] {a\\\textsc{dt}};
            \draw [thick,->] (n4) -- (n3);
            \draw [thick,->] ($(n3.east) + (1cm,0)$) -- ++(.5cm,0) node[above,midway] {\scriptsize\emph{Left}};
            \end{tikzpicture}
          \caption{}
        \end{subfigure}
      \end{column}
      \begin{column}{.5\textwidth}
        \begin{subfigure}{.5\textwidth}
          \begin{tikzpicture}[node distance=.5mm,baseline=(n1)]
            \node (n1) [notarget] {I\\\textsc{prp}};
            \node (n2) [target, right=of n1] {saw\\\textsc{vbd}};
            \node (n4) [target, below=5mm of n2] {girl\\\textsc{nn}};
            \node (n3) [notarget, below=5mm of n4] {a\\\textsc{dt}};
            \draw [thick,->] (n4) -- (n2);
            \draw [thick,->] (n3) -- (n4);
          \end{tikzpicture}
          \caption{}
        \end{subfigure}
      \end{column}
    \end{columns}
    \caption{\textsc{Izquierda}. (a) Antes. (b) Después}
  \end{figure}
\end{frame}

\subsection*{Ejemplo}

\begin{frame}[label=example]{Ejemplo}
  \framesubtitle{``\emph{Sobre la oferta de IBM}''}
  \begin{figure}[ht]
  \footnotesize
  \setlength{\extrarowheight}{16pt}
  \begin{tabular}{p{1\textwidth}}
    \begin{tikzpicture}[node distance=1mm,baseline=(sobre)]
      \node (sobre) [target] {Sobre\\\textsc{adp}};
      \node (la) [target, right=of sobre] {la\\\textsc{det}};
      \node (oferta) [notarget, right=of la] {oferta\\\textsc{noun}};
      \node (de) [notarget, right=of oferta] {de\\\textsc{adp}};
      \node (inter) [notarget, right=of de] {IBM\\\textsc{noun}};
      \node (b1) [blank,right=of inter]{};
      \draw [thick,->] (b1.west) -- ++(.5cm,0) node[above,midway]{\tiny\textsc{Shift}};
    \end{tikzpicture}
\\
   \begin{tikzpicture}[node distance=1mm,baseline=(sobre)]
     \node (sobre) [notarget] {Sobre\\\textsc{adp}};
     \node (la) [target, right=of sobre] {la\\\textsc{det}};
     \node (oferta) [target, right=of la] {oferta\\\textsc{noun}};
     \node (de) [notarget, right=of oferta] {de\\\textsc{adp}};
     \node (inter) [notarget, right=of de] {IBM\\\textsc{noun}};
     \node (b1) [blank,right=of inter]{};
   \end{tikzpicture}

  \end{tabular}
\end{figure}
\end{frame}

\begin{frame}[label=example]{Ejemplo}
  \framesubtitle{``\emph{Sobre la oferta de IBM}''}
  \begin{figure}[ht]
    \footnotesize
    \setlength{\extrarowheight}{16pt}
    \begin{tabular}{p{1\textwidth}}
    \begin{tikzpicture}[node distance=1mm,baseline=(sobre)]
      \node (sobre) [notarget] {Sobre\\\textsc{adp}};
      \node (la) [target, right=of sobre] {la\\\textsc{det}};
      \node (oferta) [target, right=of la] {oferta\\\textsc{noun}};
      \node (de) [notarget, right=of oferta] {de\\\textsc{adp}};
      \node (inter) [notarget, right=of de] {IBM\\\textsc{noun}};
      \node (b1) [blank,right=of inter]{};
      \draw [thick,->] (b1.west) -- ++(.5cm,0) node[above,midway]{\tiny\textsc{Right}};
    \end{tikzpicture}
\\
   \begin{tikzpicture}[node distance=1mm,baseline=(sobre)]
     \node (sobre) [target] {Sobre\\\textsc{adp}};
     \node (b1) [blank,right=of sobre] {};
     \node (la) [notarget,below=.3cm of oferta] {la\\\textsc{det}};
     \node (oferta) [target, right=of b1] {oferta\\\textsc{noun}};
     \node (de) [notarget, right=of oferta] {de\\\textsc{adp}};
     \node (inter) [notarget, right=of de] {IBM\\\textsc{noun}};
     \node (b1) [blank,right=of inter]{};
     \draw [thick,->] (la) -- (oferta);
   \end{tikzpicture}
    \end{tabular}
  \end{figure}
\end{frame}

\begin{frame}[label=example]{Ejemplo}
  \framesubtitle{``\emph{Sobre la oferta de IBM}''}
  \begin{figure}[ht]
    \footnotesize
    \setlength{\extrarowheight}{5pt}
    \begin{tabular}{p{1\textwidth}}
    \begin{tikzpicture}[node distance=1mm,baseline=(sobre)]
      \node (sobre) [target] {Sobre\\\textsc{adp}};
      \node (b1) [blank,right=of sobre] {};
      \node (la) [notarget,below=.3cm of oferta] {la\\\textsc{det}};
      \node (oferta) [target, right=of b1] {oferta\\\textsc{noun}};
      \node (de) [notarget, right=of oferta] {de\\\textsc{adp}};
      \node (inter) [notarget, right=of de] {IBM\\\textsc{noun}};
      \node (b1) [blank,right=of inter]{};

      \draw [thick,->] (la) -- (oferta);
      \draw [thick,->] (b1.west) -- ++(.5cm,0) node[above,midway]{\tiny\textsc{Right}};
    \end{tikzpicture}
\\
   \begin{tikzpicture}[node distance=1mm,baseline=(b2)]
     \node (b2) [blank] {};
     \node (b1) [blank,right=of b2] {};
     \node (la) [notarget,below=.3cm of oferta] {la\\\textsc{det}};
     \node (oferta) [target, right=of b1] {oferta\\\textsc{noun}};
     \node (sobre) [notarget,below=.3cm of la] {Sobre\\\textsc{adp}};
     \node (de) [target, right=of oferta] {de\\\textsc{adp}};
     \node (inter) [notarget, right=of de] {IBM\\\textsc{noun}};
     \node (b1) [blank,right=of inter]{};

     \draw [thick,->] (la) -- (oferta);
     \draw [thick,->] (sobre) -- (la);
   \end{tikzpicture}
    \end{tabular}
  \end{figure}
\end{frame}

\begin{frame}[label=example]{Ejemplo}
  \framesubtitle{``\emph{Sobre la oferta de IBM}''}
  \begin{figure}[ht]
    \footnotesize
    \setlength{\extrarowheight}{5pt}
    \begin{tabular}{p{1\textwidth}}
    \begin{tikzpicture}[node distance=1mm,baseline=(b2)]
      \node (b2) [blank] {};
      \node (b1) [blank,right=of b2] {};
      \node (la) [notarget,below=.3cm of oferta] {la\\\textsc{det}};
      \node (oferta) [target, right=of b1] {oferta\\\textsc{noun}};
      \node (sobre) [notarget,below=.3cm of la] {Sobre\\\textsc{adp}};
      \node (de) [target, right=of oferta] {de\\\textsc{adp}};
      \node (inter) [notarget, right=of de] {IBM\\\textsc{noun}};
      \node (b1) [blank,right=of inter]{};

      \draw [thick,->] (la) -- (oferta);
      \draw [thick,->] (sobre) -- (la);

      \draw [thick,->] (b1.west) -- ++(.5cm,0) node[above,midway]{\tiny\textsc{Shift}};
    \end{tikzpicture}
\\
    \begin{tikzpicture}[node distance=1mm,baseline=(b2)]
      \node (b2) [blank] {};
      \node (b1) [blank,right=of b2] {};
      \node (la) [notarget,below=.3cm of oferta] {la\\\textsc{det}};
      \node (oferta) [notarget, right=of b1] {oferta\\\textsc{noun}};
      \node (sobre) [notarget,below=.3cm of la] {Sobre\\\textsc{adp}};
      \node (de) [target, right=of oferta] {de\\\textsc{adp}};
      \node (inter) [target, right=of de] {IBM\\\textsc{noun}};
      \node (b1) [blank,right=of inter]{};

      \draw [thick,->] (la) -- (oferta);
      \draw [thick,->] (sobre) -- (la);
    \end{tikzpicture}
    \end{tabular}
  \end{figure}
\end{frame}

\begin{frame}[label=example]{Ejemplo}
  \framesubtitle{``\emph{Sobre la oferta de IBM}''}
  \begin{figure}[ht]
    \footnotesize
    \setlength{\extrarowheight}{1pt}
    \begin{tabular}{p{1\textwidth}}
    \begin{tikzpicture}[node distance=1mm,baseline=(b2)]
      \node (b2) [blank] {};
      \node (b1) [blank,right=of b2] {};
      \node (la) [notarget,below=.3cm of oferta] {la\\\textsc{det}};
      \node (oferta) [notarget, right=of b1] {oferta\\\textsc{noun}};
      \node (sobre) [notarget,below=.3cm of la] {Sobre\\\textsc{adp}};
      \node (de) [target, right=of oferta] {de\\\textsc{adp}};
      \node (inter) [target, right=of de] {IBM\\\textsc{noun}};
      \node (b1) [blank,right=of inter]{};

      \draw [thick,->] (la) -- (oferta);
      \draw [thick,->] (sobre) -- (la);

      \draw [thick,->] (b1.west) -- ++(.5cm,0) node[above,midway]{\tiny\textsc{Right}};
    \end{tikzpicture}
\\
       \begin{tikzpicture}[node distance=1mm,baseline=(b2)]
      \node (b2) [blank] {};
      \node (b1) [blank,right=of b2] {};
      \node (la) [notarget,below=.3cm of oferta] {la\\\textsc{det}};
      \node (oferta) [target, right=of b1] {oferta\\\textsc{noun}};
      \node (sobre) [notarget,below=.3cm of la] {Sobre\\\textsc{adp}};
      \node (de) [notarget, below=.3cm of inter] {de\\\textsc{adp}};
      \node (b3) [blank,right=of oferta] {};
      \node (inter) [target, right=of b3] {IBM\\\textsc{noun}};

      \draw [thick,->] (la) -- (oferta);
      \draw [thick,->] (sobre) -- (la);
      \draw [thick,->] (de) -- (inter);
    \end{tikzpicture}
    \end{tabular}
  \end{figure}
\end{frame}

\subsection*{Selección de Características}
\begin{frame}{Selección de Características}
\begin{table}[b!]
  \caption{Descripción del tipo de características y sus valores}
  
  \begin{tabularx}{\textwidth}{cp{.8\textwidth}}
    \toprule
    Tipo     & Valor                                                                \\
    \toprule
    pos      & POS \emph{tag}                                                       \\
    lex      & La palabra                                                           \\
    ch-L-pos & POS \emph{tag} del nodo hijo modificando al padre desde la izquierda \\
    ch-L-lex & Palabra del correspondiente ch-L-pos                                 \\
    ch-R-pos & POS \emph{tag} del nodo hijo que modifica al padre desde la derecha  \\
    ch-R-lex & Palabra del correspondiente ch-R-pos                                 \\
    \bottomrule
  \end{tabularx}
\end{table}  
\end{frame}


\subsection*{Resultados}


\begin{frame}{Resultados}
  \framesubtitle{Comparación de resultados}
  \begin{table}[b!]
  \begin{tabularx}{.82\textwidth}{p{.5\textwidth}|p{.1\textwidth}p{.1\textwidth}}
    Kernel: $(x'\cdot x'' + 1)^2$, Ctx: $(2,4)$ & TFG & ROHIT \\
    \toprule
    \emph{Dep. Acc.}  & 76\%   & 75\% \\
    \emph{Root Acc.}  & 67\%   & 70\% \\
    \emph{Comp. Rate} & 15\%   & 11\% \\
    \bottomrule
  \end{tabularx}
  \end{table}
\end{frame}

\section{Implementación}

\subsection*{Planificación}
\begin{frame}{Planificación}
  \definecolor{barblue}{RGB}{153,204,254}
  \definecolor{groupblue}{RGB}{51,102,254}
  \definecolor{linkred}{RGB}{165,0,33}
  \renewcommand\sfdefault{phv}
  \renewcommand\mddefault{mc}
  \renewcommand\bfdefault{bc}
  \setganttlinklabel{s-s}{START-TO-START}
  \setganttlinklabel{f-s}{FINISH-TO-START}
  \setganttlinklabel{f-f}{FINISH-TO-FINISH}
  \noindent\resizebox{\textwidth}{!}{
  \begin{tikzpicture}[x=0cm, y=0cm]
\begin{ganttchart}[
  canvas/.append style={fill=none, draw=black!5, line width=.75pt},
  hgrid style/.style={draw=black!5, line width=.75pt},
  vgrid={*1{draw=black!5, line width=.75pt}},
  today=14,
  today rule/.style={
    draw=black!64,
    dash pattern=on 3.5pt off 4.5pt,
    line width=1.5pt
  },
  today label font=\small\scshape,
  title/.style={draw=none, fill=none},
  title label font=\scshape\footnotesize,
  title label node/.append style={below=7pt},
  include title in canvas=false,
  bar label font=\mdseries\small\color{black!70},
  bar label node/.append style={left=2cm},
  bar/.append style={draw=none, fill=black!63},
  bar incomplete/.append style={fill=barblue},
  bar progress label font=\mdseries\footnotesize\color{black!70},
  group incomplete/.append style={fill=groupblue},
    group left shift=0,
    group right shift=0,
    group height=.5,
    group peaks tip position=0,
    group label node/.append style={left=.6cm},
    group progress label font=\scshape\small,
    link/.style={-latex, line width=1.5pt, linkred},
    link label font=\scriptsize\scshape,
    link label node/.append style={below left=-2pt and 0pt},
  ]{1}{14}
  [grid]
  \gantttitle{\tiny Septiembre}{4}
  \gantttitle{\tiny Octubre}{4} 
  \gantttitle{\tiny Noviembre}{4}
  \gantttitle{\tiny Diciembre}{2}\\
  \gantttitle[
    title label node/.append style={below left=7pt and -3pt}
  ]{Semana:\quad1}{1}
  \gantttitlelist{2,...,14}{1} \\
  \ganttgroup[progress=100]{Progreso}{1}{14} \\
  \ganttbar[
    progress=100,
    name=research
  ]{Investigación}{1}{4} \\
  \ganttbar[
    progress=100,
    name=design
  ]{Análisis y Diseño}{5}{5} \\
  \ganttbar[
    progress=100,
    name=impl
  ]{Implementación}{6}{11} \\
  \ganttbar[
    progress=100,
    name=memoir
  ]{Memoria}{12}{14} \\    
  
  \ganttmilestone{M1: Conocer el campo del NLP}{4}{4}  \\
  \ganttmilestone{M2: Finalizar Código}{11}{11} \\
  \ganttmilestone{M3: Finalización TFG}{14}{14}
  
  \ganttlink[link type=f-s]{research}{design}
  \ganttlink[link type=f-s]{design}{impl}
  \ganttlink[link type=f-s]{impl}{memoir}
\end{ganttchart}
\end{tikzpicture}}
\end{frame}

\begin{frame}{Implementación}
  \framesubtitle{Elección del lenguaje Scala}
  \begin{columns}[onlytextwidth,c]
    \column{.5\textwidth}
    \begin{itemize}
    \item Programación OO.
    \item Programación Funcional.
    \item Sintaxis breve.
    \item Escalable.
    \end{itemize}
    \column{.5\textwidth}
    \begin{itemize}
    \item Implementa algunos patrones.
    \item \textsc{Traits}.
    \item Amplio abanico reglas de visibilidad.
    \end{itemize}
  \end{columns}
\end{frame}

\begin{frame}[fragile]{Implementación}
  \framesubtitle{Ventajas de \textsc{Scala}}
  \begin{columns}[onlytextwidth]
    \begin{column}{.5\textwidth}
      \begin{javacode*}{fontsize=\fontsize{5.2}{4},label=\textsc{Java}}
    class WordCountMapper extends MapReduceBase
    implements Mapper<IntWritable, Text, Text, IntWritable> {

      static final IntWritable one  = new IntWritable(1);
      // Value will be set in a non-thread-safe way!
      static final Text word = new Text;

      @Override
      public void map(IntWritable key, Text valueDocContents,
      OutputCollector<Text, IntWritable> output, Reporter reporter) {
        String[] tokens = valueDocContents.toString.split("\\s+");       
        for (String wordString: tokens) {
          if (wordString.length > 0) {
            word.set(wordString.toLowerCase);
            output.collect(word, one);
          }
        }
      }
    }

    class WordCountReduce extends MapReduceBase
    implements Reducer<Text, IntWritable, Text, IntWritable> {

      public void reduce(Text keyWord, java.util.Iterator<IntWritable> counts,
      OutputCollector<Text, IntWritable> output, Reporter reporter) {
        int totalCount = 0;
        while (counts.hasNext) {                                         
          while (counts.hasNext) {
            totalCount += counts.next.get;
          }
          output.collect(keyWord, new IntWritable(totalCount));
        }
      }
      \end{javacode*}
    \end{column}
    \begin{column}{.5\textwidth}
      \begin{scalacode*}{fontsize=\fontsize{5.2}{4},label=\textsc{Scala}}
    class ScaldingWordCount(args : Args) extends Job(args) {
      TextLine(args("input"))
      .read
      .flatMap('line -> 'word) {
        line: String => line.trim.toLowerCase.split("""\s+""")
      }
      .groupBy('word){ group => group.size('count) }
      .write(Tsv(args("output")))                   
    }
      \end{scalacode*}
    \end{column}
  \end{columns}
\end{frame}

\subsection*{Metodología}
\begin{frame}{Metodología: TDD y BDD}
  \begin{columns}[onlytextwidth,c]
    \column{.5\textwidth}
    \emph{Test-Driven Development}
    \begin{itemize}
    \item Rojo.
    \item Verde.
    \item Refactorizar.
    \end{itemize}
    \column{.5\textwidth}
    \emph{Behavior-Driven Development}
    \begin{itemize}
    \item \emph{Given}.
    \item \emph{When}.
    \item \emph{Then}.
    \end{itemize}
  \end{columns}
\end{frame}

\subsection*{TDD para AA}
\begin{frame}{TDD para AA}
    TDD para AA
\end{frame}

\begin{frame}{Ejemplos de TDD}
  Poner algunas sentencias GWT    
\end{frame}

\section{Conclusiones}
\section{TRabajo futuro}

% Los diagramas de clases lo mismo no hace falta.
% Background,estado del arte.

% Van a hacer preguntas subjetivas.
% Poner en la penultima linea el github.


\end{document}